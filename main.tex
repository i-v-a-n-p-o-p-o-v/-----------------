\documentclass{article}


%% Russian language support
\usepackage{cmap}
\usepackage[T2A]{fontenc}
\usepackage[utf8]{inputenc}
\usepackage[russian]{babel}

\usepackage[a4paper]{geometry}

%% Figures
\usepackage{tkz-euclide}
\usepackage{subcaption}
\usepackage{amsmath}

%% Hyphenation rules
\usepackage{hyphenat}

%%colorfull
\usepackage{xcolor}

\hyphenation{ма-те-ма-ти-ка вос-ста-нав-ли-вать}

\title{Краткий курс геометрии если все совсем плохо}
\author{Иван Попов}

% DOCUMENT
\begin{document}

% TITLE
\pagenumbering{gobble}
\maketitle
\newpage
\pagenumbering{arabic}

% TOCS
\tableofcontents

\newpage

\section{Векторная алгебра}
\textbf{Направленный отрезок} - отрезок с указаным направлением. Направление задается при помощи точки начала и точки конца.\\
\begin{figure}[h!]
    \centering

    \begin{tikzpicture}        
        \tkzDefPoint(0,0){A} \tkzDrawPoint(A) \tkzLabelPoint[left,black](A){$A$}
        \tkzDefPoint(2,1){B} \tkzDrawPoint(B) \tkzLabelPoint[right,black](B){$B$}

        \tkzDrawSegment[-Triangle](A,B)
    \end{tikzpicture}
\caption{Направленный отрезок $\overline{AB}$}
\end{figure}
$\overline{AB} \in \overrightarrow{a}$ - направленный отрезок является представителем вектора $\overrightarrow{a}$
\\
\fbox{
    \textbf{Внимание} \underline{Направленный отрезок равен только себе}
}
\\
Совокупность напраленых отрезков является \textbf{вектором}.
\subsection{Действия над векторами и их свойства(Аксиомпатика Вейля)}
%%TODO добавить аксиоматику вейля
\subsubsection{Сложение векторов}
\paragraph{Правило треугольника}
$\\\\\overrightarrow{AC}=\overrightarrow{AB}+\overrightarrow{BC}$
\begin{figure}[h!]
    \begin{tikzpicture}        
        \tkzDefPoint(0,0){A} \tkzDrawPoint(A) \tkzLabelPoint[left,black](A){$A$}
        \tkzDefPoint(2,1){B} \tkzDrawPoint(B) \tkzLabelPoint[right,black](B){$B$}
        \tkzDefPoint(1,-1){C} \tkzDrawPoint(C) \tkzLabelPoint[right,black](C){$C$}

        \tkzDrawSegment[-Triangle](A,B)
        \tkzDrawSegment[-Triangle](B,C)
        \tkzDrawSegment[-Triangle](A,C)
    \end{tikzpicture}
\end{figure}
\paragraph{Правило параллелограма}
$\\\\\overrightarrow{AX}=\overrightarrow{AB}+\overrightarrow{AC}$
\begin{figure}[h!]
    \begin{tikzpicture}        
        \tkzDefPoint(0,0){A} \tkzDrawPoint(A) \tkzLabelPoint[left,black](A){$A$}
        \tkzDefPoint(1,2){B} \tkzDrawPoint(B) \tkzLabelPoint[left,black](B){$B$}
        \tkzDefPoint(3,0){C} \tkzDrawPoint(C) \tkzLabelPoint[right,black](C){$C$}
        \tkzDefPoint(4,2){X} \tkzDrawPoint(X) \tkzLabelPoint[right,black](X){$X$}

        \tkzDrawSegment[-Triangle](A,B)
        \tkzDrawSegment[-Triangle](A,C)
        \tkzDrawSegment[-Triangle](A,X)
        \tkzDrawSegment[-Triangle,dashed](B,X)
        \tkzDrawSegment[-Triangle,dashed](C,X)
    \end{tikzpicture}
\end{figure}
\paragraph{Правило замкнутой ломаной|многоугольника}
$\\\\\overrightarrow{AF}=\overrightarrow{AB}+\overrightarrow{BC}+\overrightarrow{CD}+\overrightarrow{DE}+\overrightarrow{EF}$
\begin{figure}[h!]
    \begin{tikzpicture}        
        \tkzDefPoint(0,0){A} \tkzDrawPoint(A) \tkzLabelPoint[left,black](A){$A$}
        \tkzDefPoint(-1,1){B} \tkzDrawPoint(B) \tkzLabelPoint[left,black](B){$B$}
        \tkzDefPoint(-1,2){C} \tkzDrawPoint(C) \tkzLabelPoint[right,black](C){$C$}
        \tkzDefPoint(0,3){D} \tkzDrawPoint(D) \tkzLabelPoint[left,black](D){$D$}
        \tkzDefPoint(2,3){E} \tkzDrawPoint(E) \tkzLabelPoint[right,black](E){$E$}
        \tkzDefPoint(0,2){F} \tkzDrawPoint(F) \tkzLabelPoint[left,black](F){$F$}

        \tkzDrawSegment[-Triangle](A,B)
        \tkzDrawSegment[-Triangle](B,C)
        \tkzDrawSegment[-Triangle](C,D)
        \tkzDrawSegment[-Triangle](D,E)
        \tkzDrawSegment[-Triangle](E,F)
        \tkzDrawSegment[-Triangle](A,F)
    \end{tikzpicture}
\end{figure}
\subsubsection{Свойства сложения векторов}
$\overrightarrow{a}+\overrightarrow{b}=\overrightarrow{b}+\overrightarrow{a}$
$\\\\\overrightarrow{a}+(\overrightarrow{b}+\overrightarrow{c})=(\overrightarrow{a}+\overrightarrow{b})+\overrightarrow{c}$
$\\\\\overrightarrow{a}+\overrightarrow{0}=\overrightarrow{0}+\overrightarrow{a}$
$\\\\\overrightarrow{\alpha}(\overrightarrow{a}+\overrightarrow{b})=\alpha*\overrightarrow{a}+\alpha*\overrightarrow{b}$
\subsubsection{Умножение вектора на число}
$k*\overrightarrow{a}=\overrightarrow{b}\\$
$k>0 => \overrightarrow{a}\uparrow\uparrow\overrightarrow{b}\\$
$k<0 => \overrightarrow{a}\uparrow\downarrow\overrightarrow{b}\\$
$|k|>1 => |\overrightarrow{a}|<|\overrightarrow{b}|\\$
$0<|k|<1 => |\overrightarrow{a}|>|\overrightarrow{b}|\\$
$k=0 => |k\overrightarrow{a}|=\overrightarrow{0}$ - нуль вектор\\
$k=1 => |\overrightarrow{a}|=|\overrightarrow{b}|\\$
\begin{figure}[h!]
    \begin{tikzpicture}        
        \tkzDefPoint(0,0){A} \tkzDrawPoint(A) \tkzLabelPoint[left,black](A){$A$}
        \tkzDefPoint(1,2){B} \tkzDrawPoint(B) \tkzLabelPoint[left,black](B){$B$}
        \tkzDefPoint(3,0){C} \tkzDrawPoint(C) \tkzLabelPoint[right,black](C){$C$}
        \tkzDefPoint(5,4){D} \tkzDrawPoint(D) \tkzLabelPoint[right,black](D){$D$}
        \tkzDefPoint(6,0){M} \tkzDrawPoint(M) \tkzLabelPoint[right,black](M){$M$}
        \tkzDefPoint(6.5,1){K} \tkzDrawPoint(K) \tkzLabelPoint[right,black](K){$K$}

        \tkzDrawSegment[-Triangle](A,B)
        \tkzDrawSegment[-Triangle](C,D)
        \tkzDrawSegment[-Triangle](K,M)

        \tkzLabelSegment[auto](A,B){1*$\overrightarrow{AB}$}
        \tkzLabelSegment[auto](C,D){2*$\overrightarrow{AB}$}
        \tkzLabelSegment[auto](M,K){-$\frac{1}{2}$*$\overrightarrow{AB}$}
    \end{tikzpicture}
\end{figure}
\subsubsection{Свойства умножения вектора на число}
k(m*$\overrightarrow{a}$)=$\overrightarrow{a}$*(k*m)=m(k*$\overrightarrow{a}$)\\
(k+m)*$\overrightarrow{a}$=k$\overrightarrow{a}$+m$\overrightarrow{a}$\\
\subsubsection{Скалярное произведение двух векторов}
Результат: скаляр\\
угол между двумя векторами\\
$\overrightarrow{a}*\overrightarrow{b}=(\overrightarrow{a},\overrightarrow{b})\\$
\\
$\overrightarrow{a}*\overrightarrow{b}=k\\$
$k>0 => \overrightarrow{a}\uparrow\uparrow\overrightarrow{b}   \angle\overrightarrow{a}\overrightarrow{b}\in[0^\circ..90^\circ)\\$
$k<0 => \overrightarrow{a}\uparrow\downarrow\overrightarrow{b}   \angle\overrightarrow{a}\overrightarrow{b}\in(90^\circ..180^\circ]\\$
$k>0 => \overrightarrow{a}\uparrow\uparrow\overrightarrow{b}\\$
$k=0 => \overrightarrow{a}||\overrightarrow{b}\in \overrightarrow{\alpha }\\$
$\overrightarrow{a}*\overrightarrow{b}=|\overrightarrow{a}|*|\overrightarrow{b}|*\cos\angle(\overrightarrow{a}\overrightarrow{b})\\$
$\cos\angle(\overrightarrow{a}\overrightarrow{b}) = \frac{\overrightarrow{a}}{|a|}*\frac{\overrightarrow{b}}{|b|}\\$
\subsubsection{Свойства скалярного произведения двух векторов}
$\overrightarrow{a}*\overrightarrow{b}=\overrightarrow{b}*\overrightarrow{a}\\$
$\overrightarrow{a}*(\overrightarrow{b}*\overrightarrow{c})=\overrightarrow{a}*\overrightarrow{b}+\overrightarrow{a}*\overrightarrow{c})\\$
$(k*\overrightarrow{a})*\overrightarrow{b}=k*(\overrightarrow{a}*\overrightarrow{b})\\$
\subsubsection{Векторое произведение двух векторов для пространства размерности 3}
Результат: вектор\\
модуль результата($\overrightarrow{c}$) равен площади параллелограма натянутого на векторы $\overrightarrow{a}$ и $\overrightarrow{b}\\$
$\overrightarrow{a}\times\overrightarrow{b}=[\overrightarrow{a}*\overrightarrow{b}]\\$
$\overrightarrow{a}\times\overrightarrow{b}=\overrightarrow{c}$
$\overrightarrow{c}\bot\overrightarrow{a},\overrightarrow{b}\\$
\subsubsection{Свойства векторного произведения двух векторов}
$\overrightarrow{a}\times\overrightarrow{b}=-\overrightarrow{b}\times\overrightarrow{a}\\$
$(\overrightarrow{a}+\overrightarrow{b})\times\overrightarrow{c}=\overrightarrow{a}\times\overrightarrow{c}+\overrightarrow{b}\times\overrightarrow{c}\\$
$(k*\overrightarrow{a})\times\overrightarrow{b}=k*(\overrightarrow{a}\times\overrightarrow{b})$
\subsubsection{Псевдоскалярное произведение двух векторов}
Результат: скаляр\\
характеризует ориентацию угла между векторами при помощи знака\\
$\overrightarrow{a}\vee\overrightarrow{b}=m\\$
$\overrightarrow{a}\vee\overrightarrow{b}=|\overrightarrow{a}|*|\overrightarrow{b}|*\sin\angle(\overrightarrow{a}\overrightarrow{b})\\$
$\sin\angle(\overrightarrow{a}\overrightarrow{b})=\frac{\overrightarrow{a}\vee\overrightarrow{b}}{|\overrightarrow{a}|*|\overrightarrow{b}|}\\$
$m=0 => \angle(\overrightarrow{a},\overrightarrow{b})=(0^\circ||180^\circ) => \overrightarrow{a}||\overrightarrow{b}\\$
\subsubsection{Свойства псевдоскалярного произведение двух векторов}
$\overrightarrow{a}\vee\overrightarrow{b}=-\overrightarrow{b}\vee\overrightarrow{a}$
$(\overrightarrow{a}+\overrightarrow{b})\vee\overrightarrow{c}=\overrightarrow{a}\vee\overrightarrow{c}+\overrightarrow{a}\vee\overrightarrow{b}\\$
$(k*\overrightarrow{a})\vee\overrightarrow{b}=k*(\overrightarrow{a}\vee\overrightarrow{b})\\$
\subsubsection{Смешаное произведение трех векторов}
Результат: скаляр\\
результат смешаного произведения представляет собой объем паралелепипеда натянутого на данные векторы\\ 
$(\overrightarrow{a}*\overrightarrow{b}*\overrightarrow{c})=\overrightarrow{a}*(\overrightarrow{b}\times\overrightarrow{c})=(\overrightarrow{a}\times\overrightarrow{b})*\overrightarrow{c}\\$
\fbox{
    \textbf{Порядок операций:  }Сначала выполняется векторное умножение ($\times$), а только затем скалярное (*)\\
}
$n=0 => \overrightarrow{a}=\overrightarrow{0}||\overrightarrow{b}=\overrightarrow{0}||\overrightarrow{c}=\overrightarrow{0}\\$
$n>0 => $Ориентация векторов такая же как в базисе $\overrightarrow{i}\overrightarrow{j}\overrightarrow{k}$\\
$n<0 => $Ориентация векторов не такая как в базисе $\overrightarrow{i}\overrightarrow{j}\overrightarrow{k}$\\
\subsubsection{Свойства смешаного произведения трех векторов}
$(\overrightarrow{a}\overrightarrow{b}\overrightarrow{c})=(\overrightarrow{b}\overrightarrow{c}\overrightarrow{a})=(\overrightarrow{c}\overrightarrow{b}\overrightarrow{b})\\$
$(\overrightarrow{a}\overrightarrow{b}\overrightarrow{c})=-(\overrightarrow{b}\overrightarrow{a}\overrightarrow{c})\\$
$((\overrightarrow{a}+\overrightarrow{b})\overrightarrow{c}\overrightarrow{d})=(\overrightarrow{a}\overrightarrow{c}\overrightarrow{d})+(\overrightarrow{b}\overrightarrow{c}\overrightarrow{d})$
\subsection{Взаимное расположение векторов, линейная зависимость и базис}
\subsubsection{Взаимное расположение векторов}
\textbf{Коллениарность} - расположение двух векторов когда они параллельны: $\overrightarrow{a}||\overrightarrow{b}$ а также $\overrightarrow{a}=k*\overrightarrow{b}$\\
\textbf{Ортогональность} - расположение двух векторов когда они перпендикулярны: $\overrightarrow{a}\perp\overrightarrow{b}\\$
\textbf{Компланарность} - расположение двух и более векторов когда они коллениарны(паралельны) одной плоскости или лежат в ней: $\overrightarrow{c}=k*\overrightarrow{a}+m*\overrightarrow{b}\\$
\subsubsection{Линейная зависимость}
\textbf{Линейная комбинация} — выражение, построенное на множестве элементов путём умножения каждого элемента на коэффициенты с последующим сложением результатов\\
$\lambda_1\overrightarrow{a_1}+\lambda\overrightarrow{a_2}+\lambda\overrightarrow{a_3}+...+\lambda\overrightarrow{a_n}=\overrightarrow{0}\\$
Линейная комбинация(Система) является линейно зависимой если хотябы 1 $\lambda\neq0$ и/или если имеется хотябы один $\overrightarrow{0}.\\$
Если система имеет линейно зависимую подсистему, то она линейно зависима.\\\\
Если мы не имеется ни одного 0, то система линейно не зависима и мы имеем размер векторного пространства $n = div(M)$
\subsubsection{Базис}
\textbf{Базис} - это упорядоченная СЛНВ (система линейно независимых векторов) в векторном пространстве.\\
Виды базисов:
\begin{itemize}
    \item Ортогональный
    \item Ортонормированый - например $(\overrightarrow{i}\overrightarrow{j}\overrightarrow{k})$
    \item Произвольный (Афинный)
\end{itemize}
\fbox{Базис позволяет определить координаты вектора}
\subsubsection{Взаимосвязь между базисами}
Пусть дан базис $\beta={\overrightarrow{e_1'},\overrightarrow{e_2'},...,\overrightarrow{e_n}'}$ и базис $\beta'=\{\overrightarrow{e_1},\overrightarrow{e_2},...,\overrightarrow{e_n}\}$, где n = dim(V)\\
Тогда координаты векторов базиса $\beta$ в базисе $\beta'$ будут представлять собой линейную комбинацию: \\
$\overrightarrow{e_1'}=a_1^1*\overrightarrow{e_1}+a_1^2*\overrightarrow{e_2}+...a_1^n*\overrightarrow{e_n}$ из чего мы получим:\\
$\overrightarrow{e_1'}\{a_1^1,a_1^2,...,a_1^n\}_\beta\\$
где $a^j_i$ - координаты\\
Формула перехода:
\fbox{
    %%TODO подозрительно
    $\overrightarrow{e_j'}=a_j^i*\overrightarrow{e_i}=\sum_{j = 1}^{n}a^j_1*\overrightarrow{e}$         $j=\overline{1,n}$
}
\paragraph*{Пример:}
$\overrightarrow{x} \in V^n$\\
$\overrightarrow{x}\{x_1,x_2,...,x_n\}_\beta$ и $\{y_1,y_2,...,y_n\}_{\beta'}$\\
$\overrightarrow{x}=y^1\overrightarrow{e_1'}+y^2\overrightarrow{e_2'}+...+y^n\overrightarrow{e_n'}=y^j\overrightarrow{e_i'}=y^1(a^i_1\overrightarrow{e_j})+y^2(a^i_2\overrightarrow{e_j})+...++y^n(a^i_n\overrightarrow{e_j}=(y^1a_1^1+y^2a_2^1+...+y^{n}a^1_n)\overrightarrow{e_1}+(y^1a_1^2+y^2a_2^2+...+y^{n}a^2_n)\overrightarrow{e_2}+...+(y^1a_1^n+y^2a_2^n+...+y^{n}a^n_n)\overrightarrow{e_n}\\$
Из этого можно сделать вывод:
$\overrightarrow{x}=x^1\overrightarrow{e_1}+x^2\overrightarrow{e_2}+...+x^n\overrightarrow{e_n}$, где $x^n=y^1a_1^n+y^2a_2^n+...+y^{n}a^n_n$
\fbox{
    $x^i=y^ja^i_j$ - формула перехода
}

%%TODO написать про нормировку
\newpage
\section{Действия над векторами в координатной форме}
Пусть даны векторы $\overrightarrow{x}\{x^1,x^2,...,x^n\}$ и $\overrightarrow{y}\{y^1,y^2,...,y^n\}\\$
\subsubsection{Сложение векторов в координатной форме}
$\overrightarrow{x}+\overrightarrow{y}=x^1\overrightarrow{x_1}+x^2\overrightarrow{x_2}+...+x^n\overrightarrow{x_n}+y^1\overrightarrow{y_1}+y^2\overrightarrow{y_2}+...+y^n\overrightarrow{y_n}=(x^1+y^1)\overrightarrow{e_1}+(x^2+y^2)\overrightarrow{e_2}+...+(x^n+y^n)\overrightarrow{e_n}=z^1\overrightarrow{e_1}+z^2\overrightarrow{e_2}+...+z^n\overrightarrow{e_n}\\$
\fbox{$x^n+y^n=z^n$}
\subsubsection{Умножение вектора на число}
$\overrightarrow{p}=k\overrightarrow{x}=k(x^1\overrightarrow{e_1}+x^2\overrightarrow{e_2}+...+x^n\overrightarrow{e_n})=kx^1\overrightarrow{e_1}+kx^2\overrightarrow{e_2}+...+kx^n\overrightarrow{e_n}$
\subsubsection{Скалярное произведение векторов}
$\overrightarrow{x}*\overrightarrow{y}=(x^1\overrightarrow{e_1}+x^2\overrightarrow{e_2}+...+x^n\overrightarrow{e_n})*(y^1\overrightarrow{e_1}+y^2\overrightarrow{e_2}+...+y^n\overrightarrow{e_n})=(x^1y^1\overrightarrow{e_1}\overrightarrow{e_1}+x^1y^2\overrightarrow{e_1}\overrightarrow{e_2}+...+x^ny^n\overrightarrow{e_n}\overrightarrow{e_n})$<= простое раскрытие произведения скобок\\
В частности для $V^3 \beta\{\overrightarrow{i},\overrightarrow{j},\overrightarrow{k}\}$ - ортогонального и ортонормированного базиса:\\
$\overrightarrow{x}*\overrightarrow{y}=(x^1\overrightarrow{i}+x^2\overrightarrow{j}+x^3\overrightarrow{k})*(y^1\overrightarrow{i}+y^2\overrightarrow{j}+y^3\overrightarrow{k})=x1y1\overrightarrow{i}^2+x^1y^2\overrightarrow{i}\overrightarrow{j}+x^1y^3\overrightarrow{i}\overrightarrow{k}+x^2y^1\overrightarrow{i}\overrightarrow{j}+x^2y^2\overrightarrow{j}^2+x^2y^3\overrightarrow{j}\overrightarrow{k}+x^3y^1\overrightarrow{i}\overrightarrow{k}+x^3y^2\overrightarrow{j}\overrightarrow{k}+x^3y^3\overrightarrow{k}^2 => x^1y^1+x^2y^2+x^3y^3\\$
Итого:
\fbox{В ортонормированом и ортогональном базисе $\overrightarrow{x}*\overrightarrow{y}=x^1y^1+x^2y^2+...+x^ny^n$}
\subsection{Псевдоскалярное произведение векторов в координатной форме в двухмерном пространстве}
$\overrightarrow{x}\{x^1,x^2\}  \overrightarrow{y}\{y^1,y^2\}\\$
$\overrightarrow{x}, \overrightarrow{y} \in \beta\{\overrightarrow{i},\overrightarrow{j}\}\\$
$\overrightarrow{x}\vee\overrightarrow{y}=x^1y^2-x^2y^1\\$
$\overrightarrow{x}^2=\overrightarrow{x}*\overrightarrow{x}=(x^1)^2+(x^2)^2\\$
\fbox{Данный вариант подходит только для пространтства размерности 2!}
\subsection{Векторное произведение двух векторов в координатной форме в трехмерном векторном простанстве}
$\beta\{\overrightarrow{i},\overrightarrow{j},\overrightarrow{k}\}\\$
$\overrightarrow{x}\times\overrightarrow{y}=$
$\begin{vmatrix}
    x^1 & x^2 & x^3\\
    y^1 & y^2 & y^3\\
    \overrightarrow{i} & \overrightarrow{j} & \overrightarrow{k}
\end{vmatrix}$
$=(x^2y^3-x^3y^2)*\overrightarrow{i}+(x^3y^1-x^1y^3)*\overrightarrow{j}+(x^1y^2-x^2y^1)*\overrightarrow{k}$
$=\{x^2y^3-x^3y^2,x^3y^1-x^1y^3,x^1y^2-x^2y^1\}$
\subsection{Смешаное произведение трех векторов в координатной форме в трехмерном векторном простанстве}
$\overrightarrow{x}\{x^1,x^2,x^3\}  \overrightarrow{y}\{y^1,y^2,y^3\}  \overrightarrow{z}\{z^1,z^2,z^3\}\\$
$(\overrightarrow{x}\overrightarrow{y}\overrightarrow{z})=(\overrightarrow{x}\times\overrightarrow{y})*\overrightarrow{z}=$
$\begin{vmatrix}
    x^1 & x^2 & x^3\\
    y^1 & y^2 & y^3\\
    z^1 & z^2 & z^3
\end{vmatrix}$
$=(x^2y^3-x^3y^2)*z^1+(x^3y^1-x^1y^3)*z^2+(x^1y^2-x^2y^1)*z^3=...$
\subsection{Векторное произведение n-1 векторов в координатной форме в n-мерном векторном простанстве}
$\beta=\{i^1,i^2,...,i^n\}, dim(V)=n\\$
$|i^k|=1, i^k \perp i^e (e \neq k)\\$
$\overrightarrow{y}=\overrightarrow{x_1}\times\overrightarrow{x_2}\times...\times\overrightarrow{x_{n-1}}=$
$\begin{vmatrix}
    x_1^1 & x_1^2 & ... & x_1^n\\
    x_2^1 & x_2^2 & ... & x_2^n\\
    ... & ... & ... & ...\\
    x_{n-1}^1 & x_{n-1}^2 & ... & x_{n-1}^n\\
    i^1 & i^2 & ... & i^n
\end{vmatrix}$ где $\overrightarrow{x_1}\{x^j_1\}$,$\overrightarrow{x_2}\{x^j_2\}$,...,$\overrightarrow{x_{n-1}}\{x^j_{n-1}\}; j=\overline{1,n}$

\subsection{Псевдоскалярное произведение n векторов в координатной форме в n-мерном векторном простанстве}
$\beta=\{i^1,i^2,...,i^n\}, dim(V)=n\\$
$|i^k|=1, i^k \perp i^e (e \neq k)\\$
$\overrightarrow{y}=\overrightarrow{x_1}\vee\overrightarrow{x_2}\vee...\vee\overrightarrow{x_{n-1}}=$
$\begin{vmatrix}
    x_1^1 & x_1^2 & ... & x_1^n\\
    x_2^1 & x_2^2 & ... & x_2^n\\
    ... & ... & ... & ...\\
    x_{n}^1 & x_{n}^2 & ... & x_{n}^n\\
\end{vmatrix}$ где $\overrightarrow{x_1}\{x^j_1\}$,$\overrightarrow{x_2}\{x^j_2\}$,...,$\overrightarrow{x_{n-1}}\{x^j_{n-1}\}; j=\overline{1,n}$
\newpage
\section{Ортогонализация и нормизация системы векторов}
Дано:\\
$\overrightarrow{a},\overrightarrow{b}\\$
Цель: найти векторы $\overrightarrow{a'}$ и $\overrightarrow{b'}$, такие что их модули равны и векторы перпендикулярны.\\
$\overrightarrow{a'},\overrightarrow{b'} : |\overrightarrow{a'}|=|\overrightarrow{b'}|=1;\overrightarrow{a'}\perp\overrightarrow{b'} \leftrightarrow \overrightarrow{a'}*\overrightarrow{b'}=0\\$
\subsection{Для двух двухмерных векторов}
$\overrightarrow{a}\{a^1,a^2\}, \overrightarrow{b}\{b^1,b^2\}\\$
\paragraph*{Шаг первый}
Определим вектор $\overrightarrow{a'}:\\$
$\overrightarrow{a'}=\overrightarrow{a}={a^1,a^2}$
\paragraph*{Шаг второй}
Определим вектор $\overrightarrow{b'}:\\$
Мы знаем что $\overrightarrow{a'}\perp\overrightarrow{b'}$, а значит мы можем воспользоваться формулой:\\
$a'^1b'^1+a'^2b'^2=0\\$
$a'^1\neq0 \Rightarrow b'^1=-\frac{a'^2}{a'^1}b'^2\\$
В итоге: $\overrightarrow{b'}=\{-\frac{a^2}{a^1}b',b'\}\\$
Как частный случай можно использовать формулу:
\fbox{
$\overrightarrow{b'}=\{-a'^2,a'^1\}$ или $\{a'^2,-a'^1\}\\$
}
\paragraph*{Шаг третий}
Проверка ориентации:\\
Если 
$det
    \begin{pmatrix}
        {a^1} & {a^2}\\
        {b^1} & {b^2}
    \end{pmatrix}
$ и $det
\begin{pmatrix}
    {a'^1} & {a'^2}\\
    {b'^1} & {b'^2}
\end{pmatrix}
$ имеют одинаковый знак, то ориентация совпала и можно переходить к нормированию. Иначе требуется вернуться на шаг 2 и выбрать другой вариант из частного случая.
\paragraph*{Нормирование}
Вектор считается нормированным, если его модуль равен 1.\\
Формула нормирования на примере вектора $\overrightarrow{a}\{a^1,a^2\}$:
\fbox{$\overrightarrow{a}=\{\frac{a^1}{\sqrt{(a^1)^2+(a^2)^2}},{\frac{a^2}{\sqrt{(a^1)^2+(a^2)^2}}}\}$}
\subsection{Для двух трехмерных векторов}
$\overrightarrow{a}\{a^1,a^2,a^3\}\\$
$\overrightarrow{b}\{b^1,b^2,b^3\}\\$
$\overrightarrow{a},\overrightarrow{b} \in V^3\\$
\paragraph*{Шаг 1}
Получим вектор $\overrightarrow{a'}$\\
$\overrightarrow{a'}=\overrightarrow{a}=\{a^1,a^2,a^3\}\\$
$\overrightarrow{a'}\perp\overrightarrow{b'}\\$
\paragraph*{Шаг 2}
Получим вектор $\overrightarrow{b'}$\\
Вектор $\overrightarrow{b'}$ является линейно зависимым для векторов $\overrightarrow{a}$ и $\overrightarrow{b}$, а значит его можно получить следующим способом:\\
$\overrightarrow{b'}=m\overrightarrow{a}+k\overrightarrow{b}={ka^1,ka^2,ka^3}+{mb^1,mb^2,mb^3}={ka^1+mb^1,ka^2+mb^2,ka^3+mb^3}\\$
Так как $\overrightarrow{a}\perp\overrightarrow{b'}$, то косинус угла между ними равен нулю, а значит $\overrightarrow{a}*\overrightarrow{b'}=0$\\
Следовательно: $a^1(ka^1+mb^1)+a^2(ka^2+mb^2)+a^3(ka^3+mb^3)=0\\$
Спустя несколько преобразований мы получим $k((a^1)^2+(a^2)^2+(a^3)^2)+m(a^1b^1+a^2b^2+a^3b^3)=0\\$\\
РЕШИМ УРАВНЕНИЕ\\
\textbf{Вариант 1}\\
$m=(a^1)^2+(a^2)^2+(a^3)^2\\$
$k=-(a^1b^1+a^2b^2+a^3b^3)\\$
\textbf{Вариант 2}\\
$m=-((a^1)^2+(a^2)^2+(a^3)^2)\\$
$k=(a^1b^1+a^2b^2+a^3b^3)\\$
\\
Заменим m и n в формуле вектора $\overrightarrow{b'}$ на полученые значения.
\paragraph*{Шаг 3}
Проверим ориентацию:
Получим векторы
$\overrightarrow{c}=\overrightarrow{a}\times\overrightarrow{b}\\$
$\overrightarrow{c'}=\overrightarrow{a'}\times\overrightarrow{b'}\\$
Проверим их коллениарность при помощи векторного произведения:\\
Если $\overrightarrow{c}\times\overrightarrow{c'}=0\\$, то переходим далее, иначе ищем ошибку в вычислениях.\\
Проверим соонаправленность векторов:
$\lambda=\frac{\overrightarrow{c}}{\overrightarrow{c'}}=\frac{c^1}{c'^1}=\frac{c^2}{c'^2}=\frac{c^3}{c'^3}\\$
Если $\lambda > 0$, тогда переходим к нормированию, иначе повторим попытку используя другой вариант из шага 2.\\
\paragraph*{Нормирование}
Формула нормирования на примере вектора $\overrightarrow{a}\{a^1,a^2,a^3\}$:\\
\fbox{$\overrightarrow{a}=\{\frac{a^1}{\sqrt{(a^1)^2+(a^2)^2+(a^3)^2}},{\frac{a^2}{\sqrt{(a^1)^2+(a^2)^2+(a^3)^2}}},{\frac{a^3}{\sqrt{(a^1)^2+(a^2)^2+(a^3)^2}}}\}$}
\subsection{Для трех трехмерных векторов}
$\overrightarrow{a}\{a^1,a^2,a^3\}\\$
$\overrightarrow{b}\{b^1,b^2,b^3\}\\$
$\overrightarrow{c}\{c^1,c^2,c^3\}\\$
$\overrightarrow{a}\perp\overrightarrow{b}\perp\overrightarrow{c}\\$
$\overrightarrow{b'}\perp\overrightarrow{c'}\\$\\
\paragraph*{Получим векторы $\overrightarrow{a'}$ и $\overrightarrow{b'}\\$}
$\overrightarrow{a'}=\overrightarrow{a}$\\
$\overrightarrow{b'}$ получаем из варианта \textbf{для двух трехмерных векторов}.\\
$\overrightarrow{c'}=\overrightarrow{a'}\times\overrightarrow{b'}$
\paragraph*{Проверим ориентацию:\\}
$\Delta1=\begin{vmatrix}
    a^1 & a^2 & a^3\\
    b^1 & b^2 & b^3\\
    c^1 & c^2 & c^3
\end{vmatrix}
$
$\Delta2=\begin{vmatrix}
    a'^1 & a'^2 & a'^3\\
    b'^1 & b'^2 & b'^3\\
    c'^1 & c'^2 & c'^3
\end{vmatrix}
\\$
Если $\Delta1$ и $\Delta2$ имеют одинаковый знак, то с ориентацией все хорошо и стоит переходить к нормированию.
\newline
\end{document}
\documentclass{article}

\usepackage[utf8]{inputenc}

%% Russian language support
\usepackage[T2A]{fontenc}
\usepackage[utf8]{inputenc}
\usepackage[russian]{babel}

\usepackage[a4paper]{geometry}

%% Figures
\usepackage{tkz-euclide}
\usepackage{subcaption}
\usepackage{amsmath}

%% Hyphenation rules
\usepackage{hyphenat}

%%colorfull
\usepackage{xcolor}

\hyphenation{ма-те-ма-ти-ка вос-ста-нав-ли-вать}

\title{Краткий курс геометрии если все совсем плохо}
\author{Иван Попов}

% DOCUMENT
\begin{document}

% TITLE
\pagenumbering{gobble}
\maketitle
\newpage
\pagenumbering{arabic}

% TOCS
\tableofcontents

\newpage

\section{Векторная алгебра}
\textbf{Направленный отрезок} - отрезок с указаным направлением. Направление задается при помощи точки начала и точки конца.\\
\begin{figure}[h!]
    \centering

    \begin{tikzpicture}        
        \tkzDefPoint(0,0){A} \tkzDrawPoint(A) \tkzLabelPoint[left,black](A){$A$}
        \tkzDefPoint(2,1){B} \tkzDrawPoint(B) \tkzLabelPoint[right,black](B){$B$}

        \tkzDrawSegment[-Triangle](A,B)
    \end{tikzpicture}
\caption{Направленный отрезок $\overline{AB}$}
\end{figure}
$\overline{AB} \in \overrightarrow{a}$ - направленный отрезок является представителем вектора $\overrightarrow{a}$
\\
\fbox{
    \textbf{Внимание} \underline{Направленный отрезок равен только себе}
}
\\
Совокупность напраленых отрезков является \textbf{вектором}.
\subsection{Действия над векторами и их свойства(Аксиомпатика Вейля)}
%%TODO добавить аксиоматику вейля
\subsubsection{Сложение векторов}
\paragraph{Правило треугольника}
$\\\\\overrightarrow{AC}=\overrightarrow{AB}+\overrightarrow{BC}$
\begin{figure}[h!]
    \begin{tikzpicture}        
        \tkzDefPoint(0,0){A} \tkzDrawPoint(A) \tkzLabelPoint[left,black](A){$A$}
        \tkzDefPoint(2,1){B} \tkzDrawPoint(B) \tkzLabelPoint[right,black](B){$B$}
        \tkzDefPoint(1,-1){C} \tkzDrawPoint(C) \tkzLabelPoint[right,black](C){$C$}

        \tkzDrawSegment[-Triangle](A,B)
        \tkzDrawSegment[-Triangle](B,C)
        \tkzDrawSegment[-Triangle](A,C)
    \end{tikzpicture}
\end{figure}
\paragraph{Правило параллелограма}
$\\\\\overrightarrow{AX}=\overrightarrow{AB}+\overrightarrow{AC}$
\begin{figure}[h!]
    \begin{tikzpicture}        
        \tkzDefPoint(0,0){A} \tkzDrawPoint(A) \tkzLabelPoint[left,black](A){$A$}
        \tkzDefPoint(1,2){B} \tkzDrawPoint(B) \tkzLabelPoint[left,black](B){$B$}
        \tkzDefPoint(3,0){C} \tkzDrawPoint(C) \tkzLabelPoint[right,black](C){$C$}
        \tkzDefPoint(4,2){X} \tkzDrawPoint(X) \tkzLabelPoint[right,black](X){$X$}

        \tkzDrawSegment[-Triangle](A,B)
        \tkzDrawSegment[-Triangle](A,C)
        \tkzDrawSegment[-Triangle](A,X)
        \tkzDrawSegment[-Triangle,dashed](B,X)
        \tkzDrawSegment[-Triangle,dashed](C,X)
    \end{tikzpicture}
\end{figure}
\paragraph{Правило замкнутой ломаной|многоугольника}
$\\\\\overrightarrow{AF}=\overrightarrow{AB}+\overrightarrow{BC}+\overrightarrow{CD}+\overrightarrow{DE}+\overrightarrow{EF}$
\begin{figure}[h!]
    \begin{tikzpicture}        
        \tkzDefPoint(0,0){A} \tkzDrawPoint(A) \tkzLabelPoint[left,black](A){$A$}
        \tkzDefPoint(-1,1){B} \tkzDrawPoint(B) \tkzLabelPoint[left,black](B){$B$}
        \tkzDefPoint(-1,2){C} \tkzDrawPoint(C) \tkzLabelPoint[right,black](C){$C$}
        \tkzDefPoint(0,3){D} \tkzDrawPoint(D) \tkzLabelPoint[left,black](D){$D$}
        \tkzDefPoint(2,3){E} \tkzDrawPoint(E) \tkzLabelPoint[right,black](E){$E$}
        \tkzDefPoint(0,2){F} \tkzDrawPoint(F) \tkzLabelPoint[left,black](F){$F$}

        \tkzDrawSegment[-Triangle](A,B)
        \tkzDrawSegment[-Triangle](B,C)
        \tkzDrawSegment[-Triangle](C,D)
        \tkzDrawSegment[-Triangle](D,E)
        \tkzDrawSegment[-Triangle](E,F)
        \tkzDrawSegment[-Triangle](A,F)
    \end{tikzpicture}
\end{figure}
\subsubsection{Свойства сложения векторов}
$\overrightarrow{a}+\overrightarrow{b}=\overrightarrow{b}+\overrightarrow{a}$
$\\\\\overrightarrow{a}+(\overrightarrow{b}+\overrightarrow{c})=(\overrightarrow{a}+\overrightarrow{b})+\overrightarrow{c}$
$\\\\\overrightarrow{a}+\overrightarrow{0}=\overrightarrow{0}+\overrightarrow{a}$
$\\\\\overrightarrow{\alpha}(\overrightarrow{a}+\overrightarrow{b})=\alpha*\overrightarrow{a}+\alpha*\overrightarrow{b}$
\subsubsection{Умножение вектора на число}
$k*\overrightarrow{a}=\overrightarrow{b}\\$
$k>0 => \overrightarrow{a}\uparrow\uparrow\overrightarrow{b}\\$
$k<0 => \overrightarrow{a}\uparrow\downarrow\overrightarrow{b}\\$
$|k|>1 => |\overrightarrow{a}|<|\overrightarrow{b}|\\$
$0<|k|<1 => |\overrightarrow{a}|>|\overrightarrow{b}|\\$
$k=0 => |k\overrightarrow{a}|=\overrightarrow{0}$ - нуль вектор\\
$k=1 => |\overrightarrow{a}|=|\overrightarrow{b}|\\$
\begin{figure}[h!]
    \begin{tikzpicture}        
        \tkzDefPoint(0,0){A} \tkzDrawPoint(A) \tkzLabelPoint[left,black](A){$A$}
        \tkzDefPoint(1,2){B} \tkzDrawPoint(B) \tkzLabelPoint[left,black](B){$B$}
        \tkzDefPoint(3,0){C} \tkzDrawPoint(C) \tkzLabelPoint[right,black](C){$C$}
        \tkzDefPoint(5,4){D} \tkzDrawPoint(D) \tkzLabelPoint[right,black](D){$D$}
        \tkzDefPoint(6,0){M} \tkzDrawPoint(M) \tkzLabelPoint[right,black](M){$M$}
        \tkzDefPoint(6.5,1){K} \tkzDrawPoint(K) \tkzLabelPoint[right,black](K){$K$}

        \tkzDrawSegment[-Triangle](A,B)
        \tkzDrawSegment[-Triangle](C,D)
        \tkzDrawSegment[-Triangle](K,M)

        \tkzLabelSegment[auto](A,B){1*$\overrightarrow{AB}$}
        \tkzLabelSegment[auto](C,D){2*$\overrightarrow{AB}$}
        \tkzLabelSegment[auto](M,K){-$\frac{1}{2}$*$\overrightarrow{AB}$}
    \end{tikzpicture}
\end{figure}
\subsubsection{Свойства умножения вектора на число}
k(m*$\overrightarrow{a}$)=$\overrightarrow{a}$*(k*m)=m(k*$\overrightarrow{a}$)\\
(k+m)*$\overrightarrow{a}$=k$\overrightarrow{a}$+m$\overrightarrow{a}$\\
\subsubsection{Скалярное произведение двух векторов}
Результат: скаляр\\
угол между двумя векторами\\
$\overrightarrow{a}*\overrightarrow{b}=(\overrightarrow{a},\overrightarrow{b})\\$
\\
$\overrightarrow{a}*\overrightarrow{b}=k\\$
$k>0 => \overrightarrow{a}\uparrow\uparrow\overrightarrow{b}   \angle\overrightarrow{a}\overrightarrow{b}\in[0^\circ..90^\circ)\\$
$k<0 => \overrightarrow{a}\uparrow\downarrow\overrightarrow{b}   \angle\overrightarrow{a}\overrightarrow{b}\in(90^\circ..180^\circ]\\$
$k>0 => \overrightarrow{a}\uparrow\uparrow\overrightarrow{b}\\$
$k=0 => \overrightarrow{a}||\overrightarrow{b}\in \overrightarrow{\alpha }\\$
$\overrightarrow{a}*\overrightarrow{b}=|\overrightarrow{a}|*|\overrightarrow{b}|*\cos\angle(\overrightarrow{a}\overrightarrow{b})\\$
$\cos\angle(\overrightarrow{a}\overrightarrow{b}) = \frac{\overrightarrow{a}}{|a|}*\frac{\overrightarrow{b}}{|b|}\\$
\subsubsection{Свойства скалярного произведения двух векторов}
$\overrightarrow{a}*\overrightarrow{b}=\overrightarrow{b}*\overrightarrow{a}\\$
$\overrightarrow{a}*(\overrightarrow{b}*\overrightarrow{c})=\overrightarrow{a}*\overrightarrow{b}+\overrightarrow{a}*\overrightarrow{c})\\$
$(k*\overrightarrow{a})*\overrightarrow{b}=k*(\overrightarrow{a}*\overrightarrow{b})\\$
\subsubsection{Векторое произведение двух векторов для пространства размерности 3}
Результат: вектор\\
модуль результата($\overrightarrow{c}$) равен площади параллелограма натянутого на векторы $\overrightarrow{a}$ и $\overrightarrow{b}\\$
$\overrightarrow{a}\times\overrightarrow{b}=[\overrightarrow{a}*\overrightarrow{b}]\\$
$\overrightarrow{a}\times\overrightarrow{b}=\overrightarrow{c}$
$\overrightarrow{c}\bot\overrightarrow{a},\overrightarrow{b}\\$
\subsubsection{Свойства векторного произведения двух векторов}
$\overrightarrow{a}\times\overrightarrow{b}=-\overrightarrow{b}\times\overrightarrow{a}\\$
$(\overrightarrow{a}+\overrightarrow{b})\times\overrightarrow{c}=\overrightarrow{a}\times\overrightarrow{c}+\overrightarrow{b}\times\overrightarrow{c}\\$
$(k*\overrightarrow{a})\times\overrightarrow{b}=k*(\overrightarrow{a}\times\overrightarrow{b})$
\subsubsection{Псевдоскалярное произведение двух векторов}
Результат: скаляр\\
характеризует ориентацию угла между векторами при помощи знака\\
$\overrightarrow{a}\vee\overrightarrow{b}=m\\$
$\overrightarrow{a}\vee\overrightarrow{b}=|\overrightarrow{a}|*|\overrightarrow{b}|*\sin\angle(\overrightarrow{a}\overrightarrow{b})\\$
$\sin\angle(\overrightarrow{a}\overrightarrow{b})=\frac{\overrightarrow{a}*\overrightarrow{b}}{|\overrightarrow{a}|*|\overrightarrow{b}|}\\$
$m=0 => \angle(\overrightarrow{a},\overrightarrow{b})=(0^\circ||180^\circ) => \overrightarrow{a}||\overrightarrow{b}\\$
\subsubsection{Свойства псевдоскалярного произведение двух векторов}
$\overrightarrow{a}\vee\overrightarrow{b}=-\overrightarrow{b}\vee\overrightarrow{a}$
$(\overrightarrow{a}+\overrightarrow{b})\vee\overrightarrow{c}=\overrightarrow{a}\vee\overrightarrow{c}+\overrightarrow{a}\vee\overrightarrow{b}\\$
$(k*\overrightarrow{a})\vee\overrightarrow{b}=k*(\overrightarrow{a}\vee\overrightarrow{b})\\$
\subsubsection{Смешаное произведение трех векторов}
Результат: скаляр\\
результат смешаного произведения представляет собой объем паралелепипеда натянутого на данные векторы\\ 
$(\overrightarrow{a}*\overrightarrow{b}*\overrightarrow{c})=\overrightarrow{a}*(\overrightarrow{b}\times\overrightarrow{c})=(\overrightarrow{a}\times\overrightarrow{b})*\overrightarrow{c}\\$
\fbox{
    \textbf{Порядок операций:  }Сначала выполняется векторное умножение ($\times$), а только затем скалярное (*)\\
}
$n=0 => \overrightarrow{a}=\overrightarrow{0}||\overrightarrow{b}=\overrightarrow{0}||\overrightarrow{c}=\overrightarrow{0}\\$
$n>0 => $Ориентация векторов такая же как в базисе $\overrightarrow{i}\overrightarrow{j}\overrightarrow{k}$\\
$n<0 => $Ориентация векторов не такая как в базисе $\overrightarrow{i}\overrightarrow{j}\overrightarrow{k}$\\
\subsubsection{Свойства смешаного произведения трех векторов}
$(\overrightarrow{a}\overrightarrow{b}\overrightarrow{c})=(\overrightarrow{b}\overrightarrow{c}\overrightarrow{a})=(\overrightarrow{c}\overrightarrow{b}\overrightarrow{b})\\$
$(\overrightarrow{a}\overrightarrow{b}\overrightarrow{c})=-(\overrightarrow{b}\overrightarrow{a}\overrightarrow{c})\\$
$((\overrightarrow{a}+\overrightarrow{b})\overrightarrow{c}\overrightarrow{d})=(\overrightarrow{a}\overrightarrow{c}\overrightarrow{d})+(\overrightarrow{b}\overrightarrow{c}\overrightarrow{d})$
\subsection{Взаимное расположение векторов, линейная зависимость и базис}
\subsubsection{Взаимное расположение векторов}
\textbf{Коллениарность} - расположение двух векторов когда они параллельны: $\overrightarrow{a}||\overrightarrow{b}$ а также $\overrightarrow{a}=k*\overrightarrow{b}$\\
\textbf{Ортогональность} - расположение двух векторов когда они перпендикулярны: $\overrightarrow{a}\perp\overrightarrow{b}\\$
\textbf{Компланарность} - расположение двух и более векторов когда они коллениарны(паралельны) одной плоскости или лежат в ней: $\overrightarrow{c}=k*\overrightarrow{a}+m*\overrightarrow{b}\\$
\subsubsection{Линейная зависимость}
\textbf{Линейная комбинация} — выражение, построенное на множестве элементов путём умножения каждого элемента на коэффициенты с последующим сложением результатов\\
$\alpha_1\overrightarrow{a_1}+\alpha_2\overrightarrow{a_2}+\alpha_3\overrightarrow{a_3}+...+\alpha_n\overrightarrow{a_n}=\overrightarrow{0}\\$
Линейная комбинация(Система) является линейно зависимой если хотябы 1 $\alpha\neq0$ и/или если имеется хотябы один $\overrightarrow{0}.\\$
Если система имеет линейно зависимую подсистему, то она линейно зависима.\\\\
Если мы не имеется ни одного 0, то система линейно не зависима и мы имеем размер векторного пространства $n = div(M)$
\subsubsection{Базис}
\textbf{Базис} - это упорядоченная СЛНВ (система линейно независимых векторов) в векторном пространстве.\\
Виды базисов:
\begin{itemize}
    \item Ортогональный
    \item Ортонормированый - например $(\overrightarrow{i}\overrightarrow{j}\overrightarrow{k})$
    \item Произвольный (Афинный)
\end{itemize}
\fbox{Базис позволяет определить координаты вектора}
\subsubsection{Взаимосвязь между базисами}
Пусть дан базис $\beta={\overrightarrow{e_1'},\overrightarrow{e_2'},...,\overrightarrow{e_n}'}$ и базис $\beta'=\{\overrightarrow{e_1},\overrightarrow{e_2},...,\overrightarrow{e_n}\}$, где n = dim(V)\\
Тогда координаты векторов базиса $\beta$ в базисе $\beta'$ будут представлять собой линейную комбинацию: \\
$\overrightarrow{e_1'}=a_1^1*\overrightarrow{e_1}+a_1^2*\overrightarrow{e_2}+...a_1^n*\overrightarrow{e_n}$ из чего мы получим:\\
$\overrightarrow{e_1'}\{a_1^1,a_1^2,...,a_1^n\}_\beta\\$
где $a^j_i$ - координаты\\
Формула перехода:
\fbox{
    %%TODO подозрительно
    $\overrightarrow{e_j'}=a_j^i*\overrightarrow{e_i}=\sum_{j = 1}^{n}a^j_1*\overrightarrow{e}$         $j=\overline{1,n}$
}
\paragraph*{Пример:}
$\overrightarrow{x} \in V^n$\\
$\overrightarrow{x}\{x_1,x_2,...,x_n\}_\beta$ и $\{y_1,y_2,...,y_n\}_{\beta'}$\\
$\overrightarrow{x}=y^1\overrightarrow{e_1'}+y^2\overrightarrow{e_2'}+...+y^n\overrightarrow{e_n'}=y^j\overrightarrow{e_i'}=y^1(a^i_1\overrightarrow{e_j})+y^2(a^i_2\overrightarrow{e_j})+...++y^n(a^i_n\overrightarrow{e_j}=(y^1a_1^1+y^2a_2^1+...+y^{n}a^1_n)\overrightarrow{e_1}+(y^1a_1^2+y^2a_2^2+...+y^{n}a^2_n)\overrightarrow{e_2}+...+(y^1a_1^n+y^2a_2^n+...+y^{n}a^n_n)\overrightarrow{e_n}\\$
Из этого можно сделать вывод:
$\overrightarrow{x}=x^1\overrightarrow{e_1}+x^2\overrightarrow{e_2}+...+x^n\overrightarrow{e_n}$, где $x^n=y^1a_1^n+y^2a_2^n+...+y^{n}a^n_n$
\fbox{
    $x^i=y^ja^i_j$ - формула перехода
}

%%TODO написать про нормировку
\end{document}
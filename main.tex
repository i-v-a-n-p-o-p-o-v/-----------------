\documentclass{book}

%% Russian language support
\usepackage{cmap}
\usepackage[T2A]{fontenc}
\usepackage[utf8]{inputenc}
\usepackage[russian]{babel}

\usepackage[a4paper]{geometry}

%% Figures
\usepackage{tkz-euclide}
\usepackage{subcaption}
\usepackage{amsmath}
\usepackage{amssymb}

%% Hyphenation rules
\usepackage{hyphenat}

\usepackage[hidelinks]{hyperref}

%%colorfull
\usepackage{xcolor}

\hyphenation{ма-те-ма-ти-ка вос-ста-нав-ли-вать}

\title{Краткий курс геометрии если все совсем плохо}
\author{Иван Попов}

% DOCUMENT
\begin{document}

% TITLE
\pagenumbering{gobble}
\maketitle
\newpage
\pagenumbering{arabic}

% TOCS
\tableofcontents

\newpage

\chapter{Векторная алгебра}
\textbf{Направленный отрезок} - отрезок с указаным направлением. Направление задается при помощи точки начала и точки конца.\\
\begin{figure}[h!]
    \centering
    \begin{tikzpicture}        
        \tkzDefPoint(0,0){A} \tkzDrawPoint(A) \tkzLabelPoint[left,black](A){$A$}
        \tkzDefPoint(2,1){B} \tkzDrawPoint(B) \tkzLabelPoint[right,black](B){$B$}

        \tkzDrawSegment[-Triangle](A,B)
    \end{tikzpicture}
\caption{Направленный отрезок $\overline{AB}$}
\end{figure}
$\overline{AB} \in \overrightarrow{a}$ - направленный отрезок является представителем вектора $\overrightarrow{a}$
\\
\fbox{
    \textbf{Внимание} \underline{Направленный отрезок равен только себе}
}
\\
Совокупность напраленых отрезков является \textbf{вектором}.
\section{Действия над векторами и их свойства(Аксиомпатика Вейля)}
%%TODO добавить аксиоматику вейля
\subsection{Сложение векторов}
\paragraph{Правило треугольника}
$\\\\\overrightarrow{AC}=\overrightarrow{A\textcolor{blue}B}+\overrightarrow{\textcolor{blue}{B}C}$
\begin{figure}[h!]
    \begin{tikzpicture}        
        \tkzDefPoint(0,0){A} \tkzDrawPoint(A) \tkzLabelPoint[left,black](A){$A$}
        \tkzDefPoint(2,1){B} \tkzDrawPoint(B) \tkzLabelPoint[right,black](B){$B$}
        \tkzDefPoint(1,-1){C} \tkzDrawPoint(C) \tkzLabelPoint[right,black](C){$C$}

        \tkzDrawSegment[-Triangle](A,B)
        \tkzDrawSegment[-Triangle](B,C)
        \tkzDrawSegment[-Triangle](A,C)
    \end{tikzpicture}
\end{figure}
\paragraph{Правило параллелограма}
$\\\\\overrightarrow{AX}=\overrightarrow{AB}+\overrightarrow{AC}$
\begin{figure}[h!]
    \begin{tikzpicture}        
        \tkzDefPoint(0,0){A} \tkzDrawPoint(A) \tkzLabelPoint[left,black](A){$A$}
        \tkzDefPoint(1,2){B} \tkzDrawPoint(B) \tkzLabelPoint[left,black](B){$B$}
        \tkzDefPoint(3,0){C} \tkzDrawPoint(C) \tkzLabelPoint[right,black](C){$C$}
        \tkzDefPoint(4,2){X} \tkzDrawPoint(X) \tkzLabelPoint[right,black](X){$X$}

        \tkzDrawSegment[-Triangle](A,B)
        \tkzDrawSegment[-Triangle](A,C)
        \tkzDrawSegment[-Triangle](A,X)
        \tkzDrawSegment[-Triangle,dashed](B,X)
        \tkzDrawSegment[-Triangle,dashed](C,X)
    \end{tikzpicture}
\end{figure}
\paragraph{Правило замкнутой ломаной|многоугольника}
$\\\\\overrightarrow{AF}=\overrightarrow{A\textcolor{blue}{B}}+\overrightarrow{\textcolor{blue}{B}\textcolor{green}{C}}+\overrightarrow{\textcolor{green}{C}\textcolor{orange}{D}}+\overrightarrow{\textcolor{orange}{D}\textcolor{pink}{E}}+\overrightarrow{\textcolor{pink}{E}F}$
\begin{figure}[h!]
    \begin{tikzpicture}        
        \tkzDefPoint(0,0){A} \tkzDrawPoint(A) \tkzLabelPoint[left,black](A){$A$}
        \tkzDefPoint(-1,1){B} \tkzDrawPoint(B) \tkzLabelPoint[left,black](B){$B$}
        \tkzDefPoint(-1,2){C} \tkzDrawPoint(C) \tkzLabelPoint[right,black](C){$C$}
        \tkzDefPoint(0,3){D} \tkzDrawPoint(D) \tkzLabelPoint[left,black](D){$D$}
        \tkzDefPoint(2,3){E} \tkzDrawPoint(E) \tkzLabelPoint[right,black](E){$E$}
        \tkzDefPoint(0,2){F} \tkzDrawPoint(F) \tkzLabelPoint[left,black](F){$F$}

        \tkzDrawSegment[-Triangle](A,B)
        \tkzDrawSegment[-Triangle](B,C)
        \tkzDrawSegment[-Triangle](C,D)
        \tkzDrawSegment[-Triangle](D,E)
        \tkzDrawSegment[-Triangle](E,F)
        \tkzDrawSegment[-Triangle](A,F)
    \end{tikzpicture}
\end{figure}
\subsection{Свойства сложения векторов}
$\overrightarrow{a}+\overrightarrow{b}=\overrightarrow{b}+\overrightarrow{a}$
$\\\\\overrightarrow{a}+(\overrightarrow{b}+\overrightarrow{c})=(\overrightarrow{a}+\overrightarrow{b})+\overrightarrow{c}$
$\\\\\overrightarrow{a}+\overrightarrow{0}=\overrightarrow{0}+\overrightarrow{a}=\overrightarrow{a}$
\subsection{Умножение вектора на число}
$k*\overrightarrow{a}=\overrightarrow{b}\\$
$k>0 => \overrightarrow{a}\uparrow\uparrow\overrightarrow{b}\\$
$k<0 => \overrightarrow{a}\uparrow\downarrow\overrightarrow{b}\\$
$|k|>1 => |\overrightarrow{a}|<|\overrightarrow{b}|\\$
$0<|k|<1 => |\overrightarrow{a}|>|\overrightarrow{b}|\\$
$k=0 => k\overrightarrow{a}=\overrightarrow{0}$ - нуль вектор\\
$k=\pm1 => |\overrightarrow{a}|=|\overrightarrow{b}|\\$
\begin{figure}[h!]
    \begin{tikzpicture}        
        \tkzDefPoint(0,0){A} \tkzDrawPoint(A) \tkzLabelPoint[left,black](A){$A$}
        \tkzDefPoint(1,2){B} \tkzDrawPoint(B) \tkzLabelPoint[left,black](B){$B$}
        \tkzDefPoint(3,0){C} \tkzDrawPoint(C) \tkzLabelPoint[right,black](C){$C$}
        \tkzDefPoint(5,4){D} \tkzDrawPoint(D) \tkzLabelPoint[right,black](D){$D$}
        \tkzDefPoint(6,0){M} \tkzDrawPoint(M) \tkzLabelPoint[right,black](M){$M$}
        \tkzDefPoint(6.5,1){K} \tkzDrawPoint(K) \tkzLabelPoint[right,black](K){$K$}

        \tkzDrawSegment[-Triangle](A,B)
        \tkzDrawSegment[-Triangle](C,D)
        \tkzDrawSegment[-Triangle](K,M)

        \tkzLabelSegment[auto](A,B){1*$\overrightarrow{AB}$}
        \tkzLabelSegment[auto](C,D){2*$\overrightarrow{AB}$}
        \tkzLabelSegment[auto](M,K){-$\frac{1}{2}$*$\overrightarrow{AB}$}
    \end{tikzpicture}
\end{figure}
\subsection{Свойства умножения вектора на число}
$(k*m)*\overrightarrow{a}=k(m*\overrightarrow{a})=m(k*\overrightarrow{a})$\\
(k+m)*$\overrightarrow{a}$=k$\overrightarrow{a}$+m$\overrightarrow{a}$\\
$\alpha*(\overrightarrow{a}+\overrightarrow{b})=\alpha*\overrightarrow{a}+\alpha*\overrightarrow{b}$

\subsection{Скалярное произведение двух векторов}
Результат: скаляр\\
Геометрический смысл: угол между двумя векторами\\
$\overrightarrow{a}*\overrightarrow{b}=(\overrightarrow{a},\overrightarrow{b})\\$
\\
$\overrightarrow{a}*\overrightarrow{b}=k=|\overrightarrow{a}|*|\overrightarrow{b}|*\cos\angle(\overrightarrow{a}\overrightarrow{b})\\$
$k>0 =>  \angle\overrightarrow{a}\overrightarrow{b}\in(0^\circ..90^\circ)\\$
$k<0 =>  \angle\overrightarrow{a}\overrightarrow{b}\in(90^\circ..180^\circ)\\$
$k>0 => \overrightarrow{a}\uparrow\uparrow\overrightarrow{b}\\$
$k=0 => \overrightarrow{a}\perp\overrightarrow{b}$ или $\overrightarrow{a}=\overrightarrow{0}$ или $\overrightarrow{b}=\overrightarrow{0}\\$
$\overrightarrow{a}*\overrightarrow{b}=|\overrightarrow{a}|*|\overrightarrow{b}|*\cos\angle(\overrightarrow{a}\overrightarrow{b})\\$
$\cos\angle(\overrightarrow{a}\overrightarrow{b}) = \frac{\overrightarrow{a}}{|a|}*\frac{\overrightarrow{b}}{|b|}\\$
\subsection{Свойства скалярного произведения двух векторов}
$\overrightarrow{a}*\overrightarrow{b}=\overrightarrow{b}*\overrightarrow{a}\\$
$\overrightarrow{a}*(\overrightarrow{b}+\overrightarrow{c})=\overrightarrow{a}*\overrightarrow{b}+\overrightarrow{a}*\overrightarrow{c}\\$
$(k*\overrightarrow{a})*\overrightarrow{b}=k*(\overrightarrow{a}*\overrightarrow{b})\\$
$\overrightarrow{a}*\overrightarrow{a}=\overrightarrow{a}^2=|\overrightarrow{a}|^2$
\subsection{Векторое произведение двух векторов для пространства размерности 3}
Результат: вектор\\
модуль результата($\overrightarrow{c}$) равен площади параллелограма натянутого на векторы $\overrightarrow{a}$ и $\overrightarrow{b}\\$
$\overrightarrow{a}\times\overrightarrow{b}=[\overrightarrow{a}*\overrightarrow{b}]\\$
$\overrightarrow{a}\times\overrightarrow{b}=\overrightarrow{c}$
$\overrightarrow{c}\bot\overrightarrow{a},\overrightarrow{b}\\$
\subsection{Свойства векторного произведения двух векторов}
$\overrightarrow{a}\times\overrightarrow{b}=-\overrightarrow{b}\times\overrightarrow{a}\\$
$(\overrightarrow{a}+\overrightarrow{b})\times\overrightarrow{c}=\overrightarrow{a}\times\overrightarrow{c}+\overrightarrow{b}\times\overrightarrow{c}\\$
$(k*\overrightarrow{a})\times\overrightarrow{b}=k*(\overrightarrow{a}\times\overrightarrow{b})\\$
$\overrightarrow{a}\times\overrightarrow{b}=\overrightarrow{0} => \overrightarrow{a}||\overrightarrow{b}$ или $\overrightarrow{a}=\overrightarrow{0}$ или $\overrightarrow{b}=\overrightarrow{0}\\$

\subsection{Псевдоскалярное произведение двух векторов}
Результат: скаляр\\
характеризует ориентацию угла между векторами при помощи знака\\
$\overrightarrow{a}\vee\overrightarrow{b}=m\\$
$\overrightarrow{a}\vee\overrightarrow{b}=|\overrightarrow{a}|*|\overrightarrow{b}|*\sin\angle(\overrightarrow{a}\overrightarrow{b})\\$
$\sin\angle(\overrightarrow{a}\overrightarrow{b})=\frac{\overrightarrow{a}\vee\overrightarrow{b}}{|\overrightarrow{a}|*|\overrightarrow{b}|}\\$

\subsection{Свойства псевдоскалярного произведение двух векторов}
$\overrightarrow{a}\vee\overrightarrow{b}=-\overrightarrow{b}\vee\overrightarrow{a}\\$
$(\overrightarrow{a}+\overrightarrow{b})\vee\overrightarrow{c}=\overrightarrow{a}\vee\overrightarrow{c}+\overrightarrow{a}\vee\overrightarrow{b}\\$
$(k*\overrightarrow{a})\vee\overrightarrow{b}=k*(\overrightarrow{a}\vee\overrightarrow{b})\\$
$\overrightarrow{a}\vee\overrightarrow{b}=0 => \overrightarrow{a}||\overrightarrow{b}$ или $\overrightarrow{a}=\overrightarrow{0}$ или $\overrightarrow{b}=\overrightarrow{0}\\$
\subsection{Смешаное произведение трех векторов}
Результат: скаляр\\
результат смешаного произведения представляет собой объем паралелепипеда натянутого на данные векторы\\ 
$(\overrightarrow{a} \overrightarrow{b} \overrightarrow{c})=\overrightarrow{a}*(\overrightarrow{b}\times\overrightarrow{c})=(\overrightarrow{a}\times\overrightarrow{b})*\overrightarrow{c}\\$
\fbox{
    \textbf{Порядок операций:  }Сначала выполняется векторное умножение ($\times$), а только затем скалярное (*)\\
}
$n=0 => \overrightarrow{a}=\overrightarrow{0}||\overrightarrow{b}=\overrightarrow{0}||\overrightarrow{c}=\overrightarrow{0}||\overrightarrow{a}=\lambda\overrightarrow{b}+\mu\overrightarrow{c}\\$
$n>0 => $Ориентация векторов такая же как в базисе $\overrightarrow{i}\overrightarrow{j}\overrightarrow{k}$\\
$n<0 => $Ориентация векторов не такая как в базисе $\overrightarrow{i}\overrightarrow{j}\overrightarrow{k}$\\
\subsection{Свойства смешаного произведения трех векторов}
$(\overrightarrow{a}\overrightarrow{b}\overrightarrow{c})=(\overrightarrow{b}\overrightarrow{c}\overrightarrow{a})=(\overrightarrow{c}\overrightarrow{a}\overrightarrow{b})\\$
$(\overrightarrow{a}\overrightarrow{b}\overrightarrow{c})=-(\overrightarrow{b}\overrightarrow{a}\overrightarrow{c})\\$
$((\overrightarrow{a}+\overrightarrow{b})\overrightarrow{c}\overrightarrow{d})=(\overrightarrow{a}\overrightarrow{c}\overrightarrow{d})+(\overrightarrow{b}\overrightarrow{c}\overrightarrow{d})$
\section{Взаимное расположение векторов, линейная зависимость и базис}
\subsection{Взаимное расположение векторов}
\textbf{Коллениарность} - расположение двух векторов когда они параллельны: $\overrightarrow{a}||\overrightarrow{b}$ а также $\overrightarrow{a}=k*\overrightarrow{b}$\\
\textbf{Ортогональность} - расположение двух векторов когда они перпендикулярны: $\overrightarrow{a}\perp\overrightarrow{b}\\$
\textbf{Компланарность} - расположение двух и более векторов когда они коллениарны(параллельны) одной плоскости или лежат в ней: $\overrightarrow{c}=k*\overrightarrow{a}+m*\overrightarrow{b}\\$
\subsection{Линейная зависимость}
\textbf{Линейная комбинация} — выражение, построенное на множестве элементов путём умножения каждого элемента на коэффициенты с последующим сложением результатов\\
$\lambda_1\overrightarrow{a_1}+\lambda_2\overrightarrow{a_2}+\lambda_3\overrightarrow{a_3}+...+\lambda_n\overrightarrow{a_n}=\overrightarrow{0}\\$
Линейная комбинация(Система) является линейно зависимой если хотябы 1 $\lambda\neq0$ и/или если имеется хотябы один $\overrightarrow{0}.\\$
Если система имеет линейно зависимую подсистему, то она линейно зависима.\\\\
Если мы не имеем ни одного 0, то система линейно не зависима и мы имеем размер векторного пространства $n = div(\overrightarrow{a_1};\overrightarrow{a_2};\overrightarrow{a_3};...;\overrightarrow{a_n})\\$
\subsection{Базис}
\textbf{Базис} - это упорядоченная СЛНВ (система линейно независимых векторов) в векторном пространстве.\\
Виды базисов:
\begin{itemize}
    \item Ортогональный
    \item Ортонормированый - например $(\overrightarrow{i}\overrightarrow{j}\overrightarrow{k})$
    \item Произвольный (Афинный)
\end{itemize}
\fbox{Базис позволяет определить координаты вектора}
\subsection{Взаимосвязь между базисами}
Пусть дан базис $\beta=\{\overrightarrow{e_1'},\overrightarrow{e_2'},...,\overrightarrow{e_n}'\}$ и базис $\beta'=\{\overrightarrow{e_1},\overrightarrow{e_2},...,\overrightarrow{e_n}\}$, где n = dim(V)\\
Тогда координаты векторов базиса $\beta$ в базисе $\beta'$ будут представлять собой линейную комбинацию: \\
$\overrightarrow{e_1'}=a_1^1*\overrightarrow{e_1}+a_1^2*\overrightarrow{e_2}+...a_1^n*\overrightarrow{e_n}$ из чего мы получим:\\
$\overrightarrow{e_1'}\{a_1^1,a_1^2,...,a_1^n\}_\beta\\$
где $a^j_i$ - координаты\\
Формула перехода:
\fbox{
    %%TODO подозрительно
    $\overrightarrow{e_j'}=a_j^i*\overrightarrow{e_i}=\sum_{i = 1}^{n}a^i_j*\overrightarrow{e_i}$         $j=\overline{1,n}$
}
\paragraph*{Пример:}
$\overrightarrow{x} \in V^n$\\
$\overrightarrow{x}\{x_1,x_2,...,x_n\}_\beta$ и $\{y_1,y_2,...,y_n\}_{\beta'}$\\
$\overrightarrow{x}=y^1\overrightarrow{e_1'}+y^2\overrightarrow{e_2'}+...+y^n\overrightarrow{e_n'}=y^j\overrightarrow{e_i'}=y^1(a^i_1\overrightarrow{e_i})+y^2(a^i_2\overrightarrow{e_i})+...+y^n(a^i_n\overrightarrow{e_i})=(y^1a_1^1+y^2a_2^1+...+y^{n}a^1_n)\overrightarrow{e_1}+(y^1a_1^2+y^2a_2^2+...+y^{n}a^2_n)\overrightarrow{e_2}+...+(y^1a_1^n+y^2a_2^n+...+y^{n}a^n_n)\overrightarrow{e_n}\\$
Из этого можно сделать вывод:
$\overrightarrow{x}=x^1\overrightarrow{e_1}+x^2\overrightarrow{e_2}+...+x^n\overrightarrow{e_n}$, где $x^i=y^1a_1^i+y^2a_2^i+...+y^{n}a^i_n$
\fbox{
    $x^i=y^ja^i_j$ - формулы перехода от старых координат к новым.
}

%%TODO написать про нормировку
%%\newpage
\chapter{Действия над векторами в координатной форме}
Пусть даны векторы $\overrightarrow{x}\{x^1,x^2,...,x^n\}$ и $\overrightarrow{y}\{y^1,y^2,...,y^n\}\\$
\subsection{Сложение векторов в координатной форме}
$\overrightarrow{x}+\overrightarrow{y}=x^1\overrightarrow{e_1}+x^2\overrightarrow{e_2}+...+x^n\overrightarrow{e_n}+y^1\overrightarrow{e_1}+y^2\overrightarrow{e_2}+...+y^n\overrightarrow{e_n}=(x^1+y^1)\overrightarrow{e_1}+(x^2+y^2)\overrightarrow{e_2}+...+(x^n+y^n)\overrightarrow{e_n}=z^1\overrightarrow{e_1}+z^2\overrightarrow{e_2}+...+z^n\overrightarrow{e_n}\\$
\fbox{$x^i+y^i=z^i; i=\overline{1,n}$}
\subsection{Умножение вектора на число}
$\overrightarrow{p}=k\overrightarrow{x}=k(x^1\overrightarrow{e_1}+x^2\overrightarrow{e_2}+...+x^n\overrightarrow{e_n})=kx^1\overrightarrow{e_1}+kx^2\overrightarrow{e_2}+...+kx^n\overrightarrow{e_n}$
\fbox{$p^i=k*x^i; i=\overline{1,n}$}
\subsection{Скалярное произведение векторов}
$\overrightarrow{x}*\overrightarrow{y}=(x^1\overrightarrow{e_1}+x^2\overrightarrow{e_2}+...+x^n\overrightarrow{e_n})*(y^1\overrightarrow{e_1}+y^2\overrightarrow{e_2}+...+y^n\overrightarrow{e_n})=(x^1y^1\overrightarrow{e_1}\overrightarrow{e_1}+x^1y^2\overrightarrow{e_1}\overrightarrow{e_2}+...+x^ny^n\overrightarrow{e_n}\overrightarrow{e_n})$<= простое раскрытие произведения скобок\\
В частности для $V^3 \beta\{\overrightarrow{i},\overrightarrow{j},\overrightarrow{k}\}$ - ортогонального и ортонормированного базиса:\\
$\overrightarrow{x}*\overrightarrow{y}=(x^1\overrightarrow{i}+x^2\overrightarrow{j}+x^3\overrightarrow{k})*(y^1\overrightarrow{i}+y^2\overrightarrow{j}+y^3\overrightarrow{k})=x^1y^1\overrightarrow{i}^2+x^1y^2\overrightarrow{i}\overrightarrow{j}+x^1y^3\overrightarrow{i}\overrightarrow{k}+x^2y^1\overrightarrow{i}\overrightarrow{j}+x^2y^2\overrightarrow{j}^2+x^2y^3\overrightarrow{j}\overrightarrow{k}+x^3y^1\overrightarrow{i}\overrightarrow{k}+x^3y^2\overrightarrow{j}\overrightarrow{k}+x^3y^3\overrightarrow{k}^2 => x^1y^1+x^2y^2+x^3y^3\\$
Итого:
\fbox{В ортонормированом и ортогональном базисе $\overrightarrow{x}*\overrightarrow{y}=x^1y^1+x^2y^2+...+x^ny^n$}
\section{Псевдоскалярное произведение векторов в координатной форме в двухмерном пространстве}
$\overrightarrow{x}\{x^1,x^2\}  \overrightarrow{y}\{y^1,y^2\}$ в $\beta\{\overrightarrow{i},\overrightarrow{j}\}$ определены своими координатами\\
$\overrightarrow{x}\vee\overrightarrow{y}=x^1y^2-x^2y^1\\$
\fbox{Данный вариант подходит только для пространтства размерности 2!}
\section{Векторное произведение двух векторов в координатной форме в трехмерном векторном простанстве}
$\beta\{\overrightarrow{i},\overrightarrow{j},\overrightarrow{k}\}\\$
$\overrightarrow{x}\times\overrightarrow{y}=$
$\begin{vmatrix}
    x^1 & x^2 & x^3\\
    y^1 & y^2 & y^3\\
    \overrightarrow{i} & \overrightarrow{j} & \overrightarrow{k}
\end{vmatrix}$
$=(x^2y^3-x^3y^2)*\overrightarrow{i}+(x^3y^1-x^1y^3)*\overrightarrow{j}+(x^1y^2-x^2y^1)*\overrightarrow{k}$
$=\{x^2y^3-x^3y^2,x^3y^1-x^1y^3,x^1y^2-x^2y^1\}$
\section{Смешаное произведение трех векторов в координатной форме в трехмерном векторном простанстве}
$\overrightarrow{x}\{x^1,x^2,x^3\}  \overrightarrow{y}\{y^1,y^2,y^3\}  \overrightarrow{z}\{z^1,z^2,z^3\}\\$
$(\overrightarrow{x}\overrightarrow{y}\overrightarrow{z})=(\overrightarrow{x}\times\overrightarrow{y})*\overrightarrow{z}=$
$\begin{vmatrix}
    x^1 & x^2 & x^3\\
    y^1 & y^2 & y^3\\
    z^1 & z^2 & z^3
\end{vmatrix}$
$=(x^2y^3-x^3y^2)*z^1+(x^3y^1-x^1y^3)*z^2+(x^1y^2-x^2y^1)*z^3=...$
\section{Векторное произведение n-1 векторов в координатной форме в n-мерном векторном простанстве}
$\beta=\{\overrightarrow{i^1},\overrightarrow{i^2},...,\overrightarrow{i^n\}}, dim(V)=n\\$
$|\overrightarrow{i^k}|=1, \overrightarrow{i^k} \perp \overrightarrow{i^e} (e \neq k)\\$
$\overrightarrow{y}=\overrightarrow{x_1}\times\overrightarrow{x_2}\times...\times\overrightarrow{x_{n-1}}=$
$\begin{vmatrix}
    x_1^1 & x_1^2 & ... & x_1^n\\
    x_2^1 & x_2^2 & ... & x_2^n\\
    ... & ... & ... & ...\\
    x_{n-1}^1 & x_{n-1}^2 & ... & x_{n-1}^n\\
    \overrightarrow{i^1} & \overrightarrow{i^2} & ... & \overrightarrow{i^n}
\end{vmatrix}$ где $\overrightarrow{x_1}\{x^j_1\}$,$\overrightarrow{x_2}\{x^j_2\}$,...,$\overrightarrow{x_{n-1}}\{x^j_{n-1}\}; j=\overline{1,n}$

\section{Смешаное произведение n векторов в координатной форме в n-мерном векторном простанстве}
$\beta=\{\overrightarrow{i^1},\overrightarrow{i^2},...,\overrightarrow{i^n\}}, dim(V)=n\\$
$|\overrightarrow{i^k}|=1, \overrightarrow{i^k} \perp \overrightarrow{i^e} (e \neq k)\\$
$\overrightarrow{y}=(\overrightarrow{x_1}\ \overrightarrow{x_2}\ ...\ \overrightarrow{x_{n}})=$
$\begin{vmatrix}
    x_1^1 & x_1^2 & ... & x_1^n\\
    x_2^1 & x_2^2 & ... & x_2^n\\
    ... & ... & ... & ...\\
    x_{n}^1 & x_{n}^2 & ... & x_{n}^n\\
\end{vmatrix}$ где $\overrightarrow{x_1}\{x^j_1\}$,$\overrightarrow{x_2}\{x^j_2\}$,...,$\overrightarrow{x_{n}}\{x^j_{n}\}; j=\overline{1,n}$
%%\newpage
\chapter{Ортогонализация и нормизация системы векторов}
Дано:\\
$\overrightarrow{a},\overrightarrow{b}\\$
Цель: найти векторы $\overrightarrow{a'}$ и $\overrightarrow{b'}$, такие что их модули равны и векторы перпендикулярны.\\
$\overrightarrow{a'},\overrightarrow{b'} : |\overrightarrow{a'}|=|\overrightarrow{b'}|=1;\overrightarrow{a'}\perp\overrightarrow{b'} \leftrightarrow \overrightarrow{a'}*\overrightarrow{b'}=0\\$
\section{Для двух двухмерных векторов}
$\overrightarrow{a}\{a^1,a^2\}, \overrightarrow{b}\{b^1,b^2\}\\$
\paragraph*{Шаг первый}
Определим вектор $\overrightarrow{a'}:\\$
$\overrightarrow{a'}=\overrightarrow{a}={a^1,a^2}$
\paragraph*{Шаг второй}
Определим вектор $\overrightarrow{b'}:\\$
Мы знаем что $\overrightarrow{a'}\perp\overrightarrow{b'}$, а значит мы можем воспользоваться формулой:\\
$a'^1b'^1+a'^2b'^2=0\\$
$a'^1\neq0 \Rightarrow b'^1=-\frac{a'^2}{a'^1}b'^2\\$
В итоге: $\overrightarrow{b'}=\{-\frac{a^2}{a^1}b',b'\}\\$
Как частный случай можно использовать формулу:
\fbox{
$\overrightarrow{b'}=\{-a'^2,a'^1\}$ или $\{a'^2,-a'^1\}\\$
}
\paragraph*{Шаг третий}
Проверка ориентации:\\
Если 
$det
    \begin{pmatrix}
        {a^1} & {a^2}\\
        {b^1} & {b^2}
    \end{pmatrix}
$ и $det
\begin{pmatrix}
    {a'^1} & {a'^2}\\
    {b'^1} & {b'^2}
\end{pmatrix}
$ имеют одинаковый знак, то ориентация совпала и можно переходить к нормированию. Иначе требуется вернуться на шаг 2 и выбрать другой вариант из частного случая.
\paragraph*{Нормирование}
Вектор считается нормированным, если его модуль равен 1.\\
Формула нормирования на примере вектора $\overrightarrow{a}\{a^1,a^2\}$:
\fbox{$\overrightarrow{a}=\{\frac{a^1}{\sqrt{(a^1)^2+(a^2)^2}},{\frac{a^2}{\sqrt{(a^1)^2+(a^2)^2}}}\}$}
\section{Для двух трехмерных векторов}
$\overrightarrow{a}\{a^1,a^2,a^3\}\\$
$\overrightarrow{b}\{b^1,b^2,b^3\}\\$
$\overrightarrow{a},\overrightarrow{b} \in V^3\\$
\paragraph*{Шаг 1}
Получим вектор $\overrightarrow{a'}$\\
$\overrightarrow{a'}=\overrightarrow{a}=\{a^1,a^2,a^3\}\\$
$\overrightarrow{a'}\perp\overrightarrow{b'}\\$
\paragraph*{Шаг 2}
Получим вектор $\overrightarrow{b'}$\\
Вектор $\overrightarrow{b'}$ является линейно зависимым для векторов $\overrightarrow{a}$ и $\overrightarrow{b}$, а значит его можно получить следующим способом:\\
$\overrightarrow{b'}=m\overrightarrow{a}+k\overrightarrow{b}={ka^1,ka^2,ka^3}+{mb^1,mb^2,mb^3}={ka^1+mb^1,ka^2+mb^2,ka^3+mb^3}\\$
Так как $\overrightarrow{a}\perp\overrightarrow{b'}$, то косинус угла между ними равен нулю, а значит $\overrightarrow{a}*\overrightarrow{b'}=0$\\
Следовательно: $a^1(ka^1+mb^1)+a^2(ka^2+mb^2)+a^3(ka^3+mb^3)=0\\$
Спустя несколько преобразований мы получим $k((a^1)^2+(a^2)^2+(a^3)^2)+m(a^1b^1+a^2b^2+a^3b^3)=0\\$\\
РЕШИМ УРАВНЕНИЕ\\
\textbf{Вариант 1}\\
$m=(a^1)^2+(a^2)^2+(a^3)^2\\$
$k=-(a^1b^1+a^2b^2+a^3b^3)\\$
\textbf{Вариант 2}\\
$m=-((a^1)^2+(a^2)^2+(a^3)^2)\\$
$k=(a^1b^1+a^2b^2+a^3b^3)\\$
\\
Заменим m и n в формуле вектора $\overrightarrow{b'}$ на полученые значения.
\paragraph*{Шаг 3}
Проверим ориентацию:
Получим векторы
$\overrightarrow{c}=\overrightarrow{a}\times\overrightarrow{b}\\$
$\overrightarrow{c'}=\overrightarrow{a'}\times\overrightarrow{b'}\\$
Проверим их коллениарность при помощи векторного произведения:\\
Если $\overrightarrow{c}\times\overrightarrow{c'}=0\\$, то переходим далее, иначе ищем ошибку в вычислениях.\\
Проверим соонаправленность векторов:
$\lambda=\frac{\overrightarrow{c}}{\overrightarrow{c'}}=\frac{c^1}{c'^1}=\frac{c^2}{c'^2}=\frac{c^3}{c'^3}\\$
Если $\lambda > 0$, тогда переходим к нормированию, иначе повторим попытку используя другой вариант из шага 2.\\
\paragraph*{Нормирование}
Формула нормирования на примере вектора $\overrightarrow{a}\{a^1,a^2,a^3\}$:\\
\fbox{$\overrightarrow{a}=\{\frac{a^1}{\sqrt{(a^1)^2+(a^2)^2+(a^3)^2}},{\frac{a^2}{\sqrt{(a^1)^2+(a^2)^2+(a^3)^2}}},{\frac{a^3}{\sqrt{(a^1)^2+(a^2)^2+(a^3)^2}}}\}$}
\section{Для трех трехмерных векторов}
$\overrightarrow{a}\{a^1,a^2,a^3\}\\$
$\overrightarrow{b}\{b^1,b^2,b^3\}\\$
$\overrightarrow{c}\{c^1,c^2,c^3\}\\$
$\overrightarrow{a}\perp\overrightarrow{b}\perp\overrightarrow{c}\\$
$\overrightarrow{b'}\perp\overrightarrow{c'}\\$\\
\paragraph*{Получим векторы $\overrightarrow{a'}$ и $\overrightarrow{b'}\\$}
$\overrightarrow{a'}=\overrightarrow{a}$\\
$\overrightarrow{b'}$ получаем из варианта \textbf{для двух трехмерных векторов}.\\
$\overrightarrow{c'}=\overrightarrow{a'}\times\overrightarrow{b'}$
\paragraph*{Проверим ориентацию:\\}
$\Delta1=\begin{vmatrix}
    a^1 & a^2 & a^3\\
    b^1 & b^2 & b^3\\
    c^1 & c^2 & c^3
\end{vmatrix}
$
$\Delta2=\begin{vmatrix}
    a'^1 & a'^2 & a'^3\\
    b'^1 & b'^2 & b'^3\\
    c'^1 & c'^2 & c'^3
\end{vmatrix}
\\$
Если $\Delta1$ и $\Delta2$ имеют одинаковый знак, то с ориентацией все хорошо и стоит переходить к нормированию.
%%\newpage
\chapter{Кординатные системы\newlineВиды и связь между ними}
\section{Декартова прямоугольная координатная система}
$V^2\ \ \ \ \ \ \ \ \beta=\{\overrightarrow{e^1},\overrightarrow{e^2}\}\\$
Координаты в декартовой системе - произведение числа на базисный вектор. К примеру координаты некоторого вектора $\overrightarrow{a} \in \beta$ будут выглядеть как:\\
$\overrightarrow{a}=\{a^1*\overrightarrow{e^1},a^2*\overrightarrow{e^2}\}$\\
Обычно мы не замечаем $\overrightarrow{e^i}$, так как мы в большинстве случаев работаем в ортонормированном базисе где они равны единице.
\begin{figure}[h!]
    \begin{tikzpicture}        
        \tkzDefPoint(-0.5,0){A} 
        \tkzDefPoint(1,0){B} \tkzDrawPoint(B) \tkzLabelPoint[right,black](B){$x$}
        \tkzDefPoint(0,-0.5){C} 
        \tkzDefPoint(0,1){D} \tkzDrawPoint(D) \tkzLabelPoint[left,black](D){$y$}
        \tkzDrawSegment[-Triangle](A,B)
        \tkzDrawSegment[-Triangle](C,D)
    \end{tikzpicture}
\end{figure}
\section{Общая декартова координатная система}
%%ПЕРЕРЫВ ДО ПОЯНЕНИЙ
%%\newpage
\chapter{Деление отрезка в заданном соотношении}
\section{На две равные части}
Дано:\\
M(x,y)\ \ \ \ \ \ \ N($\eta,\nu$)\\
|MS|=|SN|\\
Цель: найти координаты точки S\\
\fbox{
Расстояние между точками: $\rho=\sqrt{(\eta-\lambda)^2+(\nu-\mu)^2}\\$
}
$\overline{MS}=\frac{1}{2}\overline{MN}\\$
$\{s^1-\lambda;s^2-\mu\}=\frac{1}{2}\{\eta-\lambda,\nu-\mu\}\\$
\begin{equation}
    \begin{cases}
      s^1=\lambda+\frac{1}{2}(\eta-\lambda)\\
      s^2=\mu+\frac{1}{2}(\nu-\mu)
    \end{cases}
\Leftrightarrow
    \begin{cases}
        s^1=\frac{1}{2}(\eta+\lambda)\\
        s^2=\frac{1}{2}(\nu+\mu)
      \end{cases}
\end{equation}\\
Итого координаты точки:
\fbox{$S(\frac{\eta+\lambda}{2},\frac{\nu+\mu}{2})$}
\section{На две произвольные части}
Дано:\\
M(x,y)\ \ \ \ \ \ \ N($\eta,\nu$)\\
Цель: найти координаты точки S\\
\paragraph{$\overline{MS}:\overline{MN}=\frac{m}{n}\\$}
\begin{equation}
    \centering
    \begin{cases}
      s^1=\lambda+\frac{m}{m+n}(\eta-\lambda)\\
      s^2=\mu+\frac{m}{m+n}(\nu-\mu)
    \end{cases}
\end{equation}\\
\fbox{$S(\frac{m\eta+n\lambda}{m+n},\frac{m\nu+n\mu}{m+n})$}
\paragraph{$\overline{MS}:\overline{MN}=k\\$}
\fbox{$S(\frac{k\eta+\lambda}{k+1},\frac{k\nu+\mu}{k+1})$}\\
\fbox{Отрицательный результат означает что такая точка находится вне отрезка\\}
\chapter{Уравнение прямой на плоскости}
$M \in (AB) \longleftrightarrow \exists \lambda \in \mathbb{R} : AM = \lambda\overline{AB}$, где координаты точки M(x,y).
Отсюда: $\{x-a^1,y-a^2\}=\lambda\{b^1-a^1,b^2-a^2\}$
\begin{figure}[h!]
    \begin{tikzpicture}        
        \tkzDefPoint(1,0.25){A} \tkzDrawPoint(A) \tkzLabelPoint[above,black](A){$A$}
        \tkzDefPoint(3,0.75){B} \tkzDrawPoint(B) \tkzLabelPoint[above,black](B){$B$}
        \tkzDefPoint(0,0){C} 
        \tkzDefPoint(4,1){D} 
        \tkzDrawSegment(C,D)
    \end{tikzpicture}
\end{figure}
\section{Параметрическое уравнение}

\begin{equation}
    \begin{cases}
      x-a^1=\lambda(b^1-a^1)\\
      y-a^2=\lambda(b^2-a^2)
    \end{cases}
\Leftrightarrow
    \begin{cases}
      x=a^1+\lambda(b^1-a^1)\\
      y=b^1+\lambda(b^2-a^2) 
    \end{cases}
\end{equation}\\

Где $a^1$ и $b^1$ - координаты точки принадлежащей, а $(b^1-a^1)$ и $(b^2-a^2)$ координаты направляющего вектора.\\
$
    \begin{pmatrix}
        {x}\\
        {y}
    \end{pmatrix}
 = 
\begin{pmatrix}
    {a^1}\\
    {b^1}
\end{pmatrix} + \lambda
\begin{pmatrix}
    {b^1} & {-a^1}\\
    {b^2} & {-a^2}
\end{pmatrix}$\\
\section{Каноническое}
$\frac{x-a^1}{b^1-a^1}=\frac{y-a^2}{b^2-a^2}=\lambda$ При условии того что $a$ и $b$ не дают 0\\
%%$M \in (AB) \Leftrightarrow \overline{AM}||\overline{AB} \Leftrightarrow \exists \lambda \in \mathbb{R} AM=\lambda\overline{AB}$
\section{Общего вида}
$\begin{vmatrix}
    x-a^1 & b^1-a^2\\
    y-a^1 & b^2-a^2
\end{vmatrix}=0=(x-a^1)(b^2-a^2)-(b^1-a^1)(y-a^2)=xb^2-xa^2-a^1b^2+a^1a^2-b^1y+b^1a^2+a^1y-a^1a^2=(b^2-a^2)x+(a^1-b^1)y-(a^1b^2-a^2b^1)$\\
Введем обозначения:\\
$A=b^2-a^2\\$
$B=a^1-b^1\\$
$C=a^2b^1-a^1b^2\\$
Отсюда можно получить $l:Ax+By+c=0 <-$ линейное уравнение 2х неизвестных если $A \neq 0 || B \neq 0$
\section{Уравнение в отрезках}
Получаемое из общего вида при условии что $C\neq0$\\
$l:\frac{x}{a}+\frac{y}{b}=1$ или $\frac{x}{\frac{c}{a}}+\frac{y}{\frac{c}{b}}=\frac{-c}{c}$
\chapter{Способы задания прямой на плоскости}
\section{По точке и направляющему вектору или по двум точкам}
$l^A \in l,\overrightarrow{p}||l$\\
$A(a^1,a^2),B(b^1,b^2),\overrightarrow{p}\{p^1,p^2\}$ или $p\{b^1-a^1,b^2-a^2\}$
\subsection{Каноническое}
$l:\frac{x-a^1}{p^1}=\frac{y-a^2}{p^2}$
\subsection{Параметрическое}
$l:$
\begin{equation}
    \begin{cases}
      x=a^1+\lambda*p^1\\
      y=b^1+\lambda*p^2 
    \end{cases}
\end{equation}\\
\subsection{Общего вида}
Пусть:\\
$p^2=A\\$
$-p^1=B\\$
$p^1a^2-p^2a^1=C\\$
Тогда $Ax+By+C=0\\$
Или $l:\begin{vmatrix}
    x-a^1 & b^1-a^2\\
    y-a^1 & b^2-a^2
\end{vmatrix}=p^2x-p^2a^1-p^1y+p^1a^2=p^2x-p^1y+(-p^2a^1+p^1a^2)$\\
\section{По точке принадлежащей и вектору нормали}
$l^A \in l,\overrightarrow{n} \perp l$\\
$A(a^1,a^2),B(b^1,b^2),\overrightarrow{n}\{n^1,n^2\}$
\subsection{Общего вида}
$\overrightarrow{AM}*\overrightarrow{n}=0 -> \{x-a^1;y-a^2\}*\{n^1;n^2\}=0\\$
Откуда\\
$n^1(x-a^1)+n^2(y-a^2)=0\\$
$n^1x+n^2y-(n^1a^1+n^2a^2)=0\\$
$n^1x+n^2y+(-n^1a^1-n^2a^2)=0\\$
$A=n^1\\$
$B=n^2\\$
$C=-(n^1a^1+n^2a^2)\\$
\subsection{Параметрическое уравнение}
\begin{equation}
    \begin{cases}
      x-a^1=\lambda(b^1-a^1)\\
      y-a^2=\lambda(b^2-a^2)
    \end{cases}\lambda \in \mathbb{R}
\end{equation}
\subsection{Каноническое уравнение}
$l:\frac{x-a^1}{-n^2}=\frac{y-a^2}{n^1}$
\chapter{Прямая на плоскости}
$l_1: A^1x+B^1y+C^1=0\\$
$l_2: A^2x+B^2y+C^2=0\\$
\section{Взаимное расположение двух прямых на плоскости}
Существует три вида расположения двух прямых на плоскости:
\begin{itemize}
    \item $l_1||l_2$
    \item $l_1 \cap l_2$
    \item $l^1 \equiv l^2$
\end{itemize}
\paragraph*{Определить взаимное расположение можно при помощи решения системы уравнений:}
\begin{equation}
    \begin{cases}
        A^1x+B^1y+C^1=0\\
        A^2x+B^2y+C^2=0
    \end{cases}
\end{equation}
Если:
\begin{itemize}
    \item одно решение - прямые пересекаются (2)
    \item множество решений - прямые совпадают (3)
    \item нет решений - прямые параллельны (1)
\end{itemize}
\paragraph*{Определить взаимное расположение можно при помощи пропорции:}
$\frac{A^1}{A^1}=\frac{B^1}{B^2}$=$\frac{C^1}{C^2}$\\
Если:
\begin{itemize}
    \item $\frac{A^1}{A^1} \neq \frac{B^1}{B^2} \neq \frac{C^1}{C^2}$ - прямые пересекаются (2)
    \item $\frac{A^1}{A^1} = \frac{B^1}{B^2} = \frac{C^1}{C^2}$ - прямые совпадают (3)
    \item $\frac{A^1}{A^1} = \frac{B^1}{B^2} \neq \frac{C^1}{C^2}$ - прямые параллельны (1)
\end{itemize}
\paragraph*{Определить взаимное расположение можно при помощи векторов нормали:}
\begin{itemize}
    \item $\overrightarrow{n^1} \nparallel \overrightarrow{n^2}$ - прямые пересекаются (2)
    \item $\overrightarrow{n^1} || \overrightarrow{n^2}$ - прямые параллельны (1) или прямые совпадают (3)
\end{itemize}
\section{Угол между прямыми на плоскости}
$l_1: A_1x+B_1y+C_1=0 \ \ \ \ \ \overrightarrow{n_1}\{a_1,b_1\} \ \ \ \overrightarrow{p_1}\{-b_1,a_1\}$\\
$l_2: A_2x+B_2y+C_2=0 \ \ \ \ \ \overrightarrow{n_2}\{a_2,b_2\} \ \ \ \overrightarrow{p_2}\{-b_2,a_2\}$\\
\fbox{Углом между двумя прямыми считают наименьший образовавшийся}
$\varphi \in [0^\circ,90^\circ]$
$cos\angle\left\langle\overrightarrow{p^1},\overrightarrow{p^2}\right\rangle=\vert\frac{\overrightarrow{p^1}*\overrightarrow{p^2}}{|\overrightarrow{p^1}|*|\overrightarrow{p^2}|}\vert => \frac{|\overrightarrow{p^1}*\overrightarrow{p^2}|}{|\overrightarrow{p^1}|*|\overrightarrow{p^2}|}=>$
\fbox{$\frac{|a^1a^2+b^1b^2|}{\sqrt{a_1^2+b_1^2}*\sqrt{a_2^2+b_2^2}}$}=
\fbox{$\frac{|\overrightarrow{n^1}*\overrightarrow{n^2}|}{|\overrightarrow{n^1}|*|\overrightarrow{n^2}|}$}\\
$cos\angle\left\langle l^1,l^2 \right\rangle=|cos\angle\left\langle\overrightarrow{p^1},\overrightarrow{p^2}\right\rangle|$

$l1: A_1x+B_1y+C_1=0 -> \overrightarrow{n_1}\{A_1,B_1\}$\\
$l2: A_2x+B_2y+C_2=0 -> \overrightarrow{n_2}\{A_2,B_2\}$\\
$\overrightarrow{n_1}\vee\overrightarrow{n_2}=
\begin{vmatrix}
    A_1 & B_1\\
    A_2 & B_2
\end{vmatrix}=|\overrightarrow{n_1}|*|\overrightarrow{n_2}|*sin(\varphi)\\$
$sin(\varphi)=\frac{|A_1B_2-A_2B_1|}{\sqrt{A_1^2+B_1^2}*\sqrt{A_2^2+B_2^2}}\\$
$|tg(\varphi)|=|\frac{A_2B_1-A_1B_2}{B_1B_2+A_1A_2}|\\$
\section{Расстояние от точки до прямой}
$l: Ax+By+C=0\\$
$M(m_1,m_2)$
\fbox{
$\rho(M;l)=\frac{|A_1m_1+Bm_2+C|}{\sqrt{A^2+B^2}}$
}
\section{Расстояние от прямой до прямой}
Существует два варианта:
\begin{itemize}
    \item конкретное значение, если прямые параллельны
    \item неопределенное расстояние, если прямые пересекаются
\end{itemize}
\subsection{Переход к расстоянию от точки до прямой}
$l: A_1x+B_1y+C_1=0 -> \overrightarrow{n_1}\{A_1,B_1\}$\\
$m: A_2x+B_2y+C_2=0 -> \overrightarrow{n_2}\{A_2,B_2\}$\\
$M(m_1,m_2) \in m\\$
Фактически все сводится к поиску расстояния от точки, принадлежащей одной из прямых до второй прямой.
\fbox{$\rho(l;m)=\frac{|C^1-\frac{A^1}{A^2}*C^2}{\sqrt{A^2+B^2}}$}
\subsection{Частная формула для параллельных прямых}
Если преобразовать уравнение прямой m с учетом пропорциональности первых двух коэфициентов в уравнениях прямых l и m,оно примет вид $A_1+B_1+C'=0$ и мы можем использовать данное уравнение:\\
\fbox{$\rho(l;m)=\frac{|C'-C_1|}{\sqrt{A^2+B^2}}$}
\subsection{Примечания}
$a*b*sin(\phi)=|\overrightarrow{a}|*|\overrightarrow{b}|*\sqrt{1-cos^2(\phi)}|\\$
$a*b*sin(\phi)=|\overrightarrow{a}|*|\overrightarrow{b}|*\sqrt{1-(\frac{\overrightarrow{a}*\overrightarrow{b}}{|\overrightarrow{a}|*|\overrightarrow{b}|})^2}\\$
$a*b*sin(\phi)=|\overrightarrow{a}|*|\overrightarrow{b}|*\frac{\sqrt{\overrightarrow{a^2}*\overrightarrow{b^2}-(a*b)^2}}{|\overrightarrow{a}|*|\overrightarrow{b}|}|\\$
%%\newpage
\chapter{Геометрическое место точек на плоскости}
\section{Кривые второго порядка}
Уравнение кривой второго порядка выглядит следующим образом:\\
$\gamma:a_{11}x^2+2a_{12}xy+a_{22}y^2+2a_{10}x+2a_{20}y+a_{00}=0$\\
Где:\\
$a_{11}x^2+2a_{12}xy+a_{22}y^2$ - коэфициенты квадратичной формы\\
$2a_{10}x+2a_{20}y$ - линейные компоненты\\
$a_{00}$ - свободный член.\\
ранее коэфициент квадратичной формы мы встерчали в подсчете модуля вектора в афинном пространстве, и фактически он является симметрической матрицей:
$g(x;y)=a_{11}x^2+a_{12}xy+a_{22}y^2=A$
$
\begin{pmatrix}
    {a_{11}} && {a_{12}}\\
    {a_{12}} && {a_{22}}
\end{pmatrix}
$
\section{Определение типа кривой}
Для определения кривой есть два варианта:
\subsection{Первый ($a_{12}=0$) - простой}
\subsubsection{Стандартно}
Все решается путем выделения полного квадрата:
$a_{11}(x^2+2\frac{a_{10}}{a_{11}}x)+a_{22}(y^2+2\frac{a_{20}}{a_{22}}y)+a_{00}=0\\
. . . = (\frac{a_{10}}{a_{11}})^2+(\frac{a_{20}}{a_{22}})^2-a_{00}\\$
Но это сработает при условии что \fbox{$a_{11},a_{22} \neq 0$}
\subsubsection{$a_{11}=0$}
$a_{22}y^2+2a_{10}x+2a_{20}y+a_{00}=0\\$
$2a_{10}x+a_{22}(y+\frac{a_{20}}{a_{22}})^2=(\frac{a_{20}}{a_{22}})^2-a_{00}$
\subsubsection{$a_{22}=0$}
$a_{11}x^2+2a_{10}x+2a_{20}y+a_{00}=0\\$
$a_{11}(x+\frac{a_{10}}{a_{11}})^2+2a_{20}y=(\frac{a_{10}}{a_{11}})^2-a_{00}$
\subsection{Второй ($a_{12} \neq 0$) Всё плохо}
\subsubsection{Метод 1 алгебраический}
Путем перехода
$A\begin{pmatrix}
    {a_{11}} && {a_{12}}\\
    {a_{12}} && {a_{22}}
\end{pmatrix}
\leadsto  B
\begin{pmatrix}
    {b_{11}} && {0}\\
    {0} && {b_{22}}
\end{pmatrix}\\
$
$
det\begin{pmatrix}
    A-\lambda E
\end{pmatrix} = 0
$<- характеристическое уравнение\\
$
\begin{vmatrix}
    {a_{11}-\lambda} && {a_{12}}\\
    {a_{12}} && {a_{22}-\lambda}
\end{vmatrix}$
$=(a_{11}-\lambda)(a_{22}-\lambda)-a_{12}^2=a_{11}a_{22}-\lambda(a_{11}+a_{22})+\lambda^2-a_{12}^2=\lambda^2-\lambda(a_{11}+a_{22})-a_{12}^2+a_{11}a_{22}=0\\$
Решив полученое уравнение мы получим $\lambda_1$ и $\lambda_2\\$
$g'(x';y')=\lambda_1(x')^2+\lambda_2(y')^2\\$
Собственные векторы:\\
\begin{equation}\lambda_i=>
    \begin{cases}
        (a_{11}-\lambda_i)x+a_{12}y=0\\
        a_{12}x+(a_{22}-\lambda_i)y=0
    \end{cases}
\end{equation}
Решения данного уравнения дадут $\overrightarrow{n_1}\{\eta,\mu\}$ для $\lambda_1$ и $\overrightarrow{n_2}\{\phi,\psi\}$ для $\lambda_2$\\
\fbox{Тривиальное решение (нулевое) НЕ ИСПОЛЬЗУЕТСЯ!}\\
И это даст нам вектроры нового базиса, а значит получим формулу перехода:\\
\begin{equation}
    \begin{cases}
        x=\eta x'+\phi y'\\
        y=\mu x'+\psi y'
    \end{cases}
\end{equation}\\
Итого:\\
$\gamma: \lambda_1(x')^2+\lambda_2(y')^2+2(a_{10}\eta+a_{20}\mu)x'+2(a_{10}\phi+a_{20}\psi)y'+a_{00}=0\\$
Отсюда мы уже можем перейти в первый вариант\\
\chapter{Кривые второго порядка}
\section{Эллипс}
\subsection{Основные факты}
\textbf{Эллипс} - множество точек на плоскости, для каждой из которых сумма расстояний до двух фиксированных точек $F_1$ и $F_2$, называемых фокусами, является констаной, величина которой больше расстояния между фокусами.\\
Расстояние $F_1F_2$ между фокусами - фокальное расстояние. Обозначается как $2c$.\\
Если M - точка, принадлежащая эллипсу, то отрезки $F_1M$ и $F_2M$ - фокальные радиусы точки M.\\
По определению для любой точки M эллипса $F_1M+F_2M=\textit{const}$. Эту величину принято обозначать $2a$. Из определения следует что $a>c$
В прямоугольной системе координат $O_{\overrightarrow{i}\overrightarrow{j}}$,  где O - середина отрезка $F_1F_2$, а $\overrightarrow{i}\upuparrows\overrightarrow{OF_1}$, эллипс имеет уравнение:\\
\fbox{$\frac{x^2}{a^2}+\frac{y^2}{b^2}=1$ <- каноническое уравнение эллипса}
где $b^2=a^2-c^2$\\
Если $F_1$ и $F_2$ совпадают, то мы получим окружность радиуса a. В этом случае фокусы эллипса совпадают с центром окружности.
\underline{Окружность - частный случай эллипса}. При этом с=0 => a=b, и в итоге мы получаем уравнение $x^2+y^2=a^2$\\
\subsection{Геометрические свойства}
\begin{itemize}
    \item Эллипс ограничен на поскости: все его точки принадлежат прямоугольнику со сторонами которые являются касательными параллельными осям.
    \item Эллипс, заданый каноническим уравнением симметричен относительно начала координат и осей координат. Эллипс, отличный от окружности, других центров и осей симметрии не имеет. Центр симметрии является центром эллипса.
    \item Прямая, проходящая через фокусы, называется первой или фокальной осью симметрии, а перпендикую ей - второй осью симметрии. Каждая ось пересекается с эллипсом в двух точках: $A_1(a,0), A_2(-a,0), B_1(0,b), B_2(0,-b)$. Эти точки называются вершинами эллипса. Отрезки $A_1A_2 = 2a$ и $B_1B_2=2b$ называются соответственно большой и малой осями эллипса, а a и b - большой и малой полуосями эллипса.
\end{itemize}
\textbf{Эксцентриситетом} эллипса называется отношение фокального расстояния эллипса в его большей оси:\\
\fbox{$\mathcal{E}=\frac{c}{a}$ <- Эксентриситет}\\
Отсюда следует $0<\mathcal{E}<1$. При этом, если $\mathcal{E}=0$, то эллипс является окружностью. С увеличением эксцентриситета уменьшается "ширина" эллипса, он делается более "продолговатым".\\
\textbf{Дирректрисами} эллипса называются две прямые, параллельные второй оси и отстоящие от нее на расстоянии $\frac{a}{\mathcal{E}}$, где а большая(действительная полуось), а $\mathcal{E}$ - эксентриситет.\\
\fbox{$x=\pm\frac{a}{\mathcal{E}}$ <- директрисы}\\
Директрисы эллипса не пересекают его. У окрожности директрисс не существует.\\
\linebreak
\textbf{Эллипс есть множество всех точек плоскости, для каждой из которых отношение растояния до фокуса к расстоянию от этой точки есть величина постоянная, которая равна эксцентриситету}\\
\section{Гипербола}
\subsection{Основные факты}
\textbf{Гипербола} - множество точек на плоскости, для каждой из которых модуль разности расстояний до двух фиксированых точек $F_1$ и $F_2$, называемвх фокусами есть величина постоянная, меньшая, чем расстояние между фокусами.\\
Расстояние $F_1F_2$ между фокусами - фокальное расстояние, обозначается 2c.\\
Если M - точка, принадлежащая гиперболе, то отрезки $F_1M$ и $F_2M$ - фокальные радиусы точки M.\\
По определению для любой точки M эллипса $|F_1M+F_2M|=\textit{const}$. Эту величину принято обозначать $2a$. Из определения следует что $a<c$\\
В прямоугольной системе координат $O_{\overrightarrow{i}\overrightarrow{j}}$,  где O - середина отрезка $F_1F_2$, а $\overrightarrow{i}\upuparrows\overrightarrow{OF_1}$, гипербола имеет уравнение:\\
\fbox{$\frac{x^2}{a^2}-\frac{y^2}{b^2}=1$ <- каноническое уравнение эллипса}
где $b^2=c^2-a^2$\\
\subsection{Геометрические свойства}
\begin{itemize}
    \item Внутри полосы ограниченой $x=\pm a$ точек гиперболы нет. 
    \item Гипербола, заданая каноническим уравнением симметрична относительно начала координат и осей координат. Центр симметрии является центром гиперболы. ось проходящая через фокусы, называется первой или фокальной осью симметрии, а перпендикая ей - второй или мнимой осью симметрии.
    \item Фокальная ось симметрии пересекает гиперболу в двух точках $A_1(a,0), A_2(-a,0)$. Вторая ось симметрии не пересекает гиперболу. Точки $A_1$ и $A_2$ называются вершинами гиперболы, а отрезок $A_1A_2$ -  действиетльной осью. Числа a и b называются соответсвтеноо действительной и мнимой полуосями гиперболы.
\end{itemize}
\textbf{Ассимптотами} гиперболы называются прямые, к которым неограничено приблежается гипербола, при неограниченом возрастании абсцисс ее точек.\\
\fbox{$y=\pm\frac{b}{z}x$ <- уравнения асимптот}
\textbf{Эксцентриситетом} гиперболы называется отношение фокального расстояния гиперболы к ее большой оси:\\
\fbox{$\mathcal{E}=\frac{c}{a}$}\\
Отсюда следует, что $\mathcal{E}$>1\\
Чем больше эксцентриситет, тем гипербола "шире".\\
Гипербола, полуоси которой равны (a=b), называется равностононней. Ее каноническое уравнение имеет вид: $x^2-y^2=a^2$\\
Эксентриситет любой равносторонней гиперболы равен $\sqrt{2}$. Асимптотами равносторонней гиперболы являются биммектрисы координатных углов y=x и y=-x.\\
\textbf{Равносторонняя гипербола является графиком функции обратной пропорциональности.}
\textbf{Дирректрисами} гиперболы называются две прямые, параллельные второй оси и отстояшие от нее на расстоянии $\frac{a}{\mathcal{E}}$, где a - действительная полуось, а $\mathcal{E}$ - эксцентриситет.\\
\fbox{$x=\pm\frac{a}{\mathcal{E}}$ <- уравнения директрис}\\
Директрисы гиперболы не пересекают ее.
\section{Парабола}
\subsection{Основные факты}
\textbf{Параболой} называется множество всех точек плоскости, равноудаленных от фиксированной точки F, называемой фокусом, и фиксированной прямой d, называемой директрисой.\\
Расстояние от фокуса до директрисы называется фокальным параметром параболы и обозначается через p. Очевидно, p=FD, где D - проекция точки F на прямую d.\\
Если M - точка данной параболы, то отрезок FM называется фокальным радиусом точки M.\\
По определению, для любой точки M параболы $FM = \rho(M,d)$\\
В прямоугольной системе координат $O_{\overrightarrow{i}\overrightarrow{j}}$,  где O - середина отрезка $DF$, а $\overrightarrow{i}\upuparrows\overrightarrow{OF}$, парабола имеет уравнение:\\
\fbox{$y^2=2px$ <- Каноническое уравнение параболы}\\
\fbox{$x=-\frac{p}{2}$ <- Уравнение директрисы параболы}\\
Фокус имеет координаты $F(\frac{p}{2},0)$\\
\subsection{Геометрические свойства}
\begin{itemize}
    \item Все точки параболы принадлежат полуплоскости $x\geqslant0$
    \item Прямая OF является осью симметрии параболы и называется осью параболы. Центров симметрии парабола не имеет. Точка O пересечения оси с параболой называется ее вершиной.
    \item Оси выбраной системы координат имеют только одну общую точку с параболой - ее вершину. Любая другая прямая l проходящая через точку O, пересекает параболу в двух точках.
\end{itemize}   
Чем больше фокальный параметр параболы, тем больше парабола "вытянута" вдоль оси Oy.\\
Точки параболы обладают свойством, аналогичным свойству точек эллипса (гиперболы): отношение расстояний от каждой точки параболы до фокуса к расстоянию от нее до директрисы - постоянно: для параболы это отношение равно 1, по этому "единица" - эксцентриситет любой параболы.
\end{document}